\chapter{Resuelves problemas aritméticos y algebraicos}

\section{Distintas formas de números positivos}

\subsection*{Resumen}

Los números enteros son $0, 1, 2, 3, 4, 5, \ldots$ y pueden ser utilizados para
contar o ordonar objetos, por ejemplo $5$ libros.

Los números decimales son los que pueden ser escritos con cifras y coma y
pueden ser utilizado para medir distancia o otras magnitudes, por ejemplo
$5.75$ cm.

Los números racionales son de la forma $\frac{p}{q}$ donde $p, q$ son enteros
y $q \neq 0$. Son utilizadas para representar una fracción. Por ejemplo si tenemos
$2$ naranjas y $8$ manzana, tenemos $5$ frutos y $\frac{2}{10}$ son naranjas.

Los porcentajes son números decimales representando una fracción  que tiene el
número 100 como denominador. En el ejemplo precedente, $20\%$ de nuestras frutas
son naranjas.

Existen otros números positivos como $x=\sqrt{2}$ el único número tal que
$x^2 = 2$ que no pueden ser representados como racionales.

\subsection*{Ejercicio 1}

La figura siguiente representa una casa de $20$m² con tres pequeño cuartos y
un grán cuarto. Utilizar la forma de ńumero real adaptada para representar:

\begin{enumerate}
\item ¿El número de rincones?
\item ¿La superficie de un pequeño cuarto?
\item ¿El número de pequeños cuarto con respecto al ńumero total de cuartos?
\item ¿El perímetro de la casa?
\end{enumerate}

\begin{center}
\begin{tikzpicture}
  \draw (0,0) rectangle(3,3)
              rectangle(2,2)
              rectangle(1,3)
              rectangle(0,2);
  \draw (0,0) node (S) {$20$m²};
\end{tikzpicture}
\end{center}

\section{Jerarquia de las operaciones}

\subsection*{Resumen}

Podemos efectuar muchas operaciones sobre números. Deben ser hechas  en
este orden

\begin{enumerate}
\item operaciones entre paréntesis, corchetes y llaves.
\item potencias y raíces.
\item productos y cocientes.
\item sumas y restas.
\end{enumerate}

\subsection*{Ejercicio 2}

Simplificar:

\begin{enumerate}
\item $2+4\times8$, $(2+4) \times 8$, $1+7-3\times2$
\item $2^4-2$, $2^{\left(4-2\right)}$, $\left(4-2\right)^2$
\item $\sqrt{4}/2$ y $\sqrt{4/2}$
\item $\left(\left\{2+5-3\right\}\times2 \right) / \sqrt{3}$
\end{enumerate}

\subsection*{Ejercicio 3 (utilizar la calculadora)}

Señor Gonzales compra 3 kilos de papas (15.4 MXN el kilo),
6 huevos (2.5MXN el huevo) y 2 kilos de manzanas (32.65 MXN el kilo).
Tiene una reducción de 5\% sobre sus compras.

\begin{enumerate}
\item ¿Cuanto deberia  pagar sin la reducción?
\item ¿Cuanto debe realmente pagar?
\item Paga con 150MXN en liquido. ¿Cual  es su vuelto?
\end{enumerate}

\section{Porcentajes}

\subsection*{Resumen}

El porcentaje $x$\% representa la fracción $\frac{x}{100}$. Crecer $N$ un
de $x$\% significa multiplicar $N$ por $1+\frac{x}{100}$. Decrecer $N$ un
de $x$\% significa multiplicar $N$ por $1-\frac{x}{100}$.

\subsection*{Ejercicio 4}

Juan tiene quatro hermanos. Sus padres comparten 100MXN entre sus hijos.

\begin{enumerate}
\item ¿Qué porcentaje dan los padres a cada uno de sus hijos?
\item Juan ya tenía 25MXN. ¿Cuanto dinero tiene ahora?
\item Utiliza 10\% de su dinero para comprar un regalo a un amigo. ¿Cuanto
  dinero le queda?
\end{enumerate}

\subsection*{Ejercicio 5 (utilizar la calculadora)}

La población de Mexico estaba de 28296000 en 1950. Ha crecido de 119\% entre
1950 y 1975 y de 90\% entre 1975 y 2000.

\begin{enumerate}
\item Calcula la población en 1975.
\item Calcula la población en 2000.
\item Determina el porcentaje total de crecimiento entre 1950 y 2000.
\end{enumerate}

\section{Soluciones de los ejercicios}

\subsection*{Ejercicio 1}

\begin{enumerate}
\item La casa tiene $4\times4=16$ rincones.
\item Un pequeño cuarto tiene una superficie de $\frac{20}{9}$ m²
\item $\frac{3}{4} = 0.75 = 75\%$ de los cuartos son pequeños.
\item El lado del grán cuadrado es $\sqrt{20}$m, entonces el perímetro de la 
  casa es $4 \sqrt{20}$.
\end{enumerate}

\subsection*{Ejercicio 2}

Simplificar:

\begin{enumerate}
\item $34$, $48$ y $2$
\item $14$, $4$ y $4$
\item $1$ y $\sqrt{2}$
\item $\frac{8}{\sqrt{3}} = \frac{8 \sqrt{3}}{3}$
\end{enumerate}

\subsection*{Ejercicio 3}

\begin{enumerate}
\item $3\times15.4 + 6 * 2.5 + 2 \times 32.65 = 126.5$ MXN.
\item Debe pagar solo $100 - 5 = 95\%$ es decir $95 \times 126.5 / 100 = 120.175$ MXN
\item $150 - 120.175 = 29.825$ MXN
\end{enumerate}

\subsection*{Ejercicio 4}

\begin{enumerate}
\item $\frac{1}{5} = \frac{20}{100} = 20\%$
\item $25+20=45$MXN
\item $45 \times \left(1 -  \frac{10}{100}\right) = 40.5$ MXN.
\end{enumerate}

\subsection*{Ejercicio 5 (utilizar la calculadora)}

\begin{enumerate}
\item $28296000 \times 2.19 = 61968240$
\item $61968240 \times 1.90 = 117739656$
\item $\frac{117739656}{28296000} = 4.161$ entonces la población ha crecido de
  $316.1\%$. Eso es diferente de $119 + 90 = 209$!
\end{enumerate}
