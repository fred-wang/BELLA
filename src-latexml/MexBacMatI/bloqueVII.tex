\chapter{Resuelve ecuaciones lineales II}

\section{Sistema de ecuaciones con dos incógnitas}

\subsection*{Resumen}

Un sistema de ecuaciones (linaeles) con dos incógnitas es un conjunto finito
de ecuación $ax + by = c$ donde $x, y$ son incógnitas y $a,b,c$ constantes. Un
sistema de $n$ ecuaciones se puede escribir

$\left\{
\begin{aligned}
  a_1 x + b_1 y = c_1 \\
  a_2 x + b_2 y = c_2 \\
  \ldots \\
  a_n x + b_n y = c_n  
\end{aligned}\right.$

Podemos efectuar muchas operaciones para tratar de resolver un sistema. Las
operaciones deben ser invertibles para conservar la equivalencia con el sistema
original.

\begin{enumerate}
\item Tomar la suma de dos ecuaciones para obtener otra ecuación:
${(a_1+a_2)} x + {(b_1+b_2)} y = c_1 + c_2$.
\item O la resta: ${(a_1-a_2)} x + {(b_1-b_2)} y = c_1 - c_2$.
\item Multiplicar/Dividir por una ecuación coeficiente $k$:
  $k a_1 x + k b_1 y = k c_1$.
\item Realizar una substitución. Por ejemplo, si $a_1 = 1$,
  $x = c_1 - b_1 y$ y haciendo esta substitución en la secunda ecuación,
  obtenemos $a_2 c_1 + (b_2 - a_2 b_1) y  = c_2$.
\end{enumerate}

Suponemos que el sistema comporte sólo una ecuación $ax+by=c$.
Si $a = b = c = 0$, todos los $(x,y) \in {\mathbb R}^2$ son soluciones.
Si $a = b = 0$ y $c \neq 0$ no hay solución. Si $b \neq 0$, obtenemos la
ecuación $y = -\frac{a}{c}x +\frac{c}{b}$ entonces hay una infinitad de
soluciones
$(x, y)$ para cada $x \in {\mathbb R}$. De la misma manera, si $a \neq 0$, hay 
una infinitad de soluciones $({-\frac{b}{a}y+\frac{c}{a}}, y)$
para cada $y \in {\mathbb R}$.

Ahora, consideramos dos ecuaciones $ax+by=k$ y $cx+dy=l$. Multiplicando
la primera equación por $c$ y la segunda por $a$, obtenemos

$\left\{
\begin{aligned}
  ca x + cb y = ck \\
  ac x + ad y = al \\
\end{aligned}\right.$

y si tomamos la resta de los dos obtenemos
${(ad - bc)} y = ck - al$. Entonces, si el determinante
$\Delta = ad - bc$ no es cero obtenemos $y = \frac{al - ck}{\Delta}$. De la
misma manera, obtenemos $x = \frac{dk - bl}{\Delta}$ y entonces el sistema
tiene una solución $(x, y)$. Si $\Delta = 0$, $ad = bc$ y 
$al = a{(cx+dy)} = acx + ady = acx+bcy = c{(ax+by)} = ck$ y de la misma manera
$bl = dk$. Entonces si $a \neq 0$, las soluciones son la de la ecucación 
$ax + by = k$ si $l = -\frac{ck}{a}$ y si no, no hay solución. Si $b \neq 0$,
las soluciones son la de la ecucación $ax + by = k$ si $l = -\frac{dk}{b}$
y si no, no hay solución. Si $a = b = 0$, las soluciones son la de la ecucación
$cx + dy = l$ si $k = 0$ y si no, no hay solución.

Podemos aplicar esos metodos para resolver un sistema de $n \geq 3$ ecuaciones.
En general, dos ecuaciones determinan una única solución $(x,y)$ y entonces
un sistema de $n \geq 3$ ecuaciones no tiene solución si esas soluciones no
son compatibles.

\subsection*{Ejemplo}

\begin{enumerate}
\item 
$\left\{\begin{aligned}
  4x + 3y = 0 \\
  z = x + y
\end{aligned}\right.$ no es un sistema de dos ecuaciones (tiene tres
incógnitas).

\item 
$\left\{\begin{aligned}
  x^2 + 2y = 1 \\
  x + \sqrt{y} = -2
\end{aligned}\right.$ no es un sistema linear (contene $x^2, \sqrt{y}$)

\item
$\left\{
\begin{aligned}
  4x + 3y = 2x - 7 \\
  3x = 2x + 3y
\end{aligned}\right.$ es un sistema de dos ecuaciones, equivalente a la forma
simplificada
$\left\{
\begin{aligned}
  2x+3y = -7 \\
  x -4y = 0
\end{aligned}\right.$
\item $2x + y = 1$ tiene una infinitad de solución
  $\{ {(x, y)} : x \in \mathbb R, y = 1 - 2x \}$
\item $5x = 2$ tiene una infinitad de solución
  $\{ {(\frac{2}{5}, y)} : y \in {\mathbb R} \}$
\item  Para resolver $\left\{
\begin{aligned}
  2x+3y = -7 \\
  x -4y = 0
\end{aligned}\right.$ podemos hacer la subsitución $x = 4y$ en la primera
ecuación: $8y + 3y = -7$ es decir
$y = -\frac{7}{11}$ y $x = 4y = -\frac{28}{11}$.
\item  Para resolver $\left\{
\begin{aligned}
  2x+3y = -7 \\
  x -4y = 0
\end{aligned}\right.$ podemos multiplicar la segunda ecuacíon por 2:
$2x - 8y = 0$ y tomar la resta con la primera:
$-7 - 0 = {(2x + 3y)} - {(2x-8y)} = 11y$. Entonces $y = -\frac{7}{11}$ y
$x = 4y = -\frac{28}{11}$.
\item  Para resolver $\left\{
\begin{aligned}
  2x+3y = -7 \\
  x -4y = 0
\end{aligned}\right.$ podemos calcular el determinante
$\Delta = 2 \times -4 - 3 \times 1 = -11$ y utilizar las formulas
$x = \frac{-4 \times -7 - 2 \times 0}{-11} = -\frac{28}{11}$ y
$y = \frac{2 \times 0 - 1 \times -7}{-11} = -\frac{7}{11}$.
\item  El determinante de $\left\{
\begin{aligned}
  -21x+6y = 3 \\
  7x-2y = 2
\end{aligned}\right.$ es $\Delta = -21 \times -2 - 6 \times 7 = 0$. Si
multiplicamos la primera por $-\frac{1}{3}$, obtenemos $7x-2y=-1$ entonces
no es compatible con la secunda ecuación $7x-2y = 2$ y el sistema no tiene
solución.
\item En el sistema de tres ecuaciones $\left\{
\begin{aligned}
  -4x-6y = 14 \\
  x -4y = 0 \\
  2x+3y = -7
\end{aligned}\right.$ la primera ecuacion es equivalente a la tercera y
entonces se reduce al sistema de dos ecuaciones $\left\{\begin{aligned}
  2x+3y = -7 \\
  x -4y = 0
\end{aligned}\right.$ de solución $y = -\frac{7}{11}$ y
$x = -\frac{28}{7}$.
\item $y = -\frac{7}{11}$ y $x = -\frac{28}{11}$ es la única solucíon de las
dos últimas ecuaciones de $\left\{
\begin{aligned}
  22y = -14 \\
  x -4y = 0 \\
  2x+3y = -7
\end{aligned}\right.$ y $22 \times -\frac{7}{11} = -14$ entonces es también la
solución del sistema de tres ecuaciones.

\item $y = -\frac{7}{11}$ y $x = -\frac{28}{11}$ es la única solucíon de las
dos últimas ecuaciones de $\left\{
\begin{aligned}
  11x = 2 \\
  x -4y = 0 \\
  2x+3y = -7
\end{aligned}\right.$ pero
$11 \times -\frac{28}{11} = -28 \neq 2$ entonces el sistema de tres ecuaciones
no tiene solución.

\end{enumerate}

\subsection*{Ejercicio 1}

Resolver las sistemas siguientes con dos incógnitas $x,y$:

\begin{enumerate}
\item $x + 2 - 9y = 8 - 3y + 2x$
\item $\left\{\begin{aligned}
  3x + 2y = 8 \\
  y - 7 = 2y + 3
\end{aligned}\right.$
\item $\left\{\begin{aligned}
   7x  - 8y = 4 \\
  -3x + 5y = 7
\end{aligned}\right.$

\item $\left\{\begin{aligned}
   21x  - 14y = 4 \\
  -3x + 2y = 7
\end{aligned}\right.$

\item $\left\{\begin{aligned}
   3x-2y=2+6y \\
  4x-8y=4-2x+8y
\end{aligned}\right.$

\item $\left\{\begin{aligned}
   2x-7y=9 \\
  2x=2-9y \\
  4x-10y-9=9+4y
\end{aligned}\right.$

\item $\left\{\begin{aligned}
   7x  - 8y = 4 \\
  -3x + 5y = 7 \\
  x - y = 0
\end{aligned}\right.$

\end{enumerate}

\subsection*{Ejercicio 2}

Determinar el determinante de esos sistemas y deducir las soluciones.

\begin{enumerate}
\item $\left\{\begin{aligned}
  3x+y=-2 \\
  27x-9y=0
\end{aligned}\right.$
\item $\left\{\begin{aligned} 3x+y=-2\\ 27x+9y=0\end{aligned}\right.$
\item $\left\{\begin{aligned} 8x+y=64\\ \frac{x}{2}+\frac{y}{16}=4\end{aligned}\right.$
\end{enumerate}

\section{Interpretación gráfica}

\subsection*{Resumen}

Como hemos visto en el capítulo precedente,
si $a,b \neq 0$, las soluciones de la ecuación $ax+by=c$ en el plano
${\mathbb R}^2$ son los puntos de la recta pasando por $(0, -\frac{c}{b})$ y
$(-\frac{c}{a}, 0)$. Si $a = 0$ y $b \neq 0$, es la recta pasando por
$(0, -\frac{c}{b})$ y paralela a eje $Y$. Si $a = 0$ y $b \neq 0$, es la recta
pasando por $(-\frac{c}{a}, 0)$ y paralela a eje $X$. Finalmente, si
$a = b = 0$ es todo el plano (si $c = 0$) o el conjunto vació ($c \neq 0$).

Para un sistema de dos ecuaciones, tomemos la interseción de los conjuntos.
En general (si ningúno es vacío o todo el plano) tenemos dos rectas. Si las
ecuaciones son incompatibles, las rectas son paralelas y la interseción es
vacía. De otra manera, hay una única solución que es la interseción de las dos
rectas.

Para un sistema de $n \geq 3$ ecuaciones, en general tenemos tres $n$ rectas
que tienen poco suerte que intersectarse.

\subsection*{Ejemplo}

Podemos ver las rectas representando las soluciones de las ecuaciones
$x=-4$, $y=-5$, $x-2y=6$, $x-2y=0$, $5x-4y=0$. Las rectas $x=-4$ y $y = -5$ son
paralela a los ejes. Las rectas de $x - 2y$ y $x - 2y = 6$ son paralelas y
entonces el sistema $\left\{\begin{aligned}
   x-2y=0 \\
  x-2y=6
\end{aligned}\right.$ no tiene solución. Las rectas de $5x - 4y = 0$ y
$x - 2y = 6$ se intersectan en $(-4, -5)$ y entonces es la única solución del
sistema $\left\{\begin{aligned}
   5x-4y=0 \\
  x-2y=6
\end{aligned}\right.$

\begin{center}
  \begin{tikzpicture}[domain=-10:10, xscale=.5, yscale=.5]
    \draw[->] (-11,0) -- (11,0) node[right] {$x$}; 
    \draw[->] (0,-11) -- (0,11) node[above] {$y$};
    \draw[color=blue] plot (\x,\x/2-3) node[left] {$x - 2y = 6$};
    \draw[color=red] (-10,-5)--(8,-5) node[above] {$y = -5$};
    \draw[color=green] (-4,-10)--(-4,10) node[left] {$x = -4$};
    \draw[color=orange] plot (\x,\x/2) node[left] {$x - 2y = 0$};
    \draw[color=purple] plot (\x,5*\x/4) node[left] {$5x - 4y = 0$};

    \foreach \x in {-10,-5,5,10}
      \draw (\x,-.1) --(\x,.1) node[above] {$\x$};
    \foreach \y in {-10,-5,5,10}
      \draw (-.1,\y) --(.1,\y) node[left] {$\y$};
  \end{tikzpicture}
\end{center}

\subsection*{Ejercicio 3}

Graficar las rectas de ecuaciones $3x-y=5$ y $2x-3y=7$. Deducir la solución
del sistema $\left\{\begin{aligned}
   3x-y=5 \\
  2x-3y=7
\end{aligned}\right.$

\subsection*{Ejercicio 4 (problema)}

Juan vende $1$ litro de leche $15$MXN y la comida de sus vacas le cuesta
$24$ MXN más $3$MXN para cada litro.
Representar el costo de $x$ litros y de la comida de las
vacas sobre un gráfico, para $0 \leq x \leq 5$. Deducir el mínimo de leche
que debe vender para obtener un beneficio positivo.

\section{Soluciones de los ejercicios}

\subsection*{Ejercicio 1}

\begin{enumerate}
\item La ecuación se simplifia en $y = \frac{x}{6} - 1$. Hay una infinitad de
  solución ${(x, \frac{x}{6} - 1)}$
\item $\left\{\begin{aligned}
  3x + 2y = 8 \\
  y - 7 = 2y + 3
\end{aligned}\right.$ La secunda ecuación se simplifia en $y = -10$.
  Entonces la primera devene $3x - 20 = 8$ y $x = \frac{28}{3}$.
\item $x = \frac{76}{11}$, $y = \frac{61}{11}$
\item No tiene solución.
\item Una infinidad de soluciones $x=\frac{8y+2}{3}$, $y \in {\mathbb R}$.
\item $x = \frac{95}{32}$, $y = -\frac{7}{16}$
\item No tiene solución.
\end{enumerate}

\subsection*{Ejercicio 2}

\begin{enumerate}
\item $\Delta = 3 \times -9 - 1 \times 27 = -54$, $x=-\frac{1}{3}$,
  $y = -1$
\item $\Delta = 3 \times 9 - 1 \times 27 = 0$. La secunda ecuación
  se simplifia en $3x+y=0$ y no es compatible con la primera. No hay solución.
\item $\Delta = 8 \times \frac{1}{16} - \frac{1}{2} = 0$. La secunda ecuación
  es un múltiplo de la primera. Hay una infinidad de soluciones
  $y \in \mathbb R$ y $x = -\frac{y}{8} - 8$.
\end{enumerate}

\subsection*{Ejercicio 3}

Las rectas se intersectan en ${(x,y)}={(2,1)}$ que es la única solución del
sistema.

\begin{center}
  \begin{tikzpicture}[domain=-10:10, xscale=.5, yscale=.5]
    \draw[->] (-10,0) -- (10,0) node[right] {$x$}; 
    \draw[->] (0,-10) -- (0,10) node[above] {$y$};
    \draw[color=blue] plot (\x,7/3-2*\x/3) node[left] {$2x - 3y = 7$};
    \draw[color=purple] plot [domain=-2:5] (\x,3*\x-5) node[left] {$3x-y =5$};

    \foreach \x in {-7.5,-5,-4,-3,-2,-1,1,2,3,4,5,7.5}
      \draw (\x,-.1) --(\x,.1) node[below] {$\x$};
    \foreach \y in {-7.5,-5,-4,-3,-2,-1,1,2,3,4,5,7.5}
      \draw (-.1,\y) --(.1,\y) node[left] {$\y$};

    \draw[style=dashed,color=red] (2,0) -- (2,1) -- (0,1); 
  \end{tikzpicture}
\end{center}

\subsection*{Ejercicio 4}

El costo de $x$ litros es $y_1 = 15x$ y el de la comida de vacas es
$y_2 = 24 + 3x$. El beneficio es positivo si $y_1 \geq y_2$ es decir cuando
la recta $y_1 = 15x$ es arriba de la recta $y_2 = 24 + 3x$. Es el caso si
$x \geq 2$.

\begin{center}
  \begin{tikzpicture}[domain=0:3, xscale=1, yscale=.1]
    \draw[->] (0,0) -- (6,0) node[right] {$x$}; 
    \draw[->] (0,0) -- (0,50) node[above] {$y$};
    \draw[color=blue] plot (\x,15*\x) node[left] {$y_1=15x$};
    \draw[color=orange] plot (\x,3*\x+24) node[left] {$y_2 =3x+24$};
    \draw[style=dashed,color=red] (2,0) -- (2,30); 

    \foreach \x in {1,2,3,4,5}
      \draw (\x,-.1) --(\x,.1) node[above] {$\x$};
    \foreach \y in {10,20,30,40}
      \draw (-.1,\y) --(.1,\y) node[left] {$\y$};
  \end{tikzpicture}
\end{center}
