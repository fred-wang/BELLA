\chapter{Realizas transformaciones algebraicas II}

\section{Técnicas de factorización}

\subsection*{Resumen}

Ya hemos visto productos notables (factor común, cuadrado de un binomio,
cubo de un binomio, binomios conjugados). Otra técnica para factorizar un
polinomio es encontra un ``cero trivial'' $a$, para deducir un factor
$X - a$. Podemos también tratar escribir el polinomio como un producto. Por
ejemplo si queremos factorizar el polinomio $P = x^2 - 3x + 2$ como un
producto de dos factores lineales,
${(x - x_1)}{(x-x_2)} = x^2 - {(x_1+x_2)}x + x_1 x_2$ y podemos poner
$x_1 = 1$ y $x_2 = 2$.

\subsection*{Ejercico 1}

\begin{enumerate}
\item Factorizar $X^2 + 34X + 289$ (producto notable)

\item Sea $P = 6 Y^2 + Y -35$. Factorizar $P$ como un producto
  ${(aY + b)}{(cY + d)}$ donde $a, b, c, d$ son enteros

\item Sea $P = X^3+33X^2+255x-289$. Calcular $P$ en $x = -1, 0, 1$. Factorizar
  $P$.
\end{enumerate}

\subsection*{Ejercico 2}

Consideramos $P = Z^2 + \left(\frac{5}{2} - \sqrt{3}\right) Z -
\frac{5 \sqrt{3}}{2}$.

\begin{enumerate}
\item Developar $Q = \left( Z + \frac{5}{4} - \frac{\sqrt{3}}{2} \right)^2$
\item Mostrar que $k = Q - P$ es una constante.
\item Encontrar enteros $a, b$ tales que
  $\left(\frac{a \sqrt{3} + b}{4}\right)^2 = k$.
\item Utilizar un producto notable para factorizar $P = Q - k$.
\end{enumerate}

\section{Expresiones racionales y división de polinomio}

\subsection*{Resumen}

Una expresión racional se escribe como una fración de dos polinomios
$\frac{P}{Q}$. Si $P, Q$ tienen un factor comúm, podemos simplifiar la fracción
por este factor.

Podemos también definir la división de dos polinomios $A$ 
y $B$ de la misma manera como la división Euclídea de dos enteros:

$$A = B Q + R$$

donde el cociente $Q$ y el resto $R$ son únicos polinomios tales que el grado
de $R$ es estrictamente menor que el grado de $B$. En particular, podemos
escribir la egualidad de expresiónes racionales:

$$\frac{A}{B} = Q + \frac{R}{B}$$

\subsection*{Ejemplo}

\begin{enumerate}
\item $\frac{X^3-1}{X^2-X}$ es una expresión racional. El numerador es
  $X^3 - 1 = {(X-1)}{(X^2+X+1)}$ y el denominador $X^2+X = X{(X-1)}$. Entonces
  podemos simplificar la expresión racional como $\frac{X^2+X+1}{X}$.
\item Tenemos $X^2+X+1 = Q X + R$ donde $Q = X + 1$ y $R = 1$. El grado de $R$
  es estrictamente menor que el grado de $X$ entonces eso es la divisíon de
  $X^2+X+1$ por $X$. La expresión racional precedente puede ser escrita
  $X+1 + \frac{1}{X}$.
\end{enumerate}

\subsection*{Ejercicio 3}

Efectuar la división de los polinomios siguientes:

\begin{enumerate}
\item $X^5+2X^3-X^2+X+2$ por $X^2+2$
\item $X^8-2X^7+X^6+X^2-4X+3$ por $X^7 - X^6$
\item $X^5+16X^4+64X^3-X^2-16X-64$ y $X^4+17X^3+80X^2+64X$
\end{enumerate}

\subsection*{Ejercicio 4}

Simplificar las expresión racionales siguientes:

\begin{enumerate}
\item $\frac{X^5+2X^3-X^2+X+2}{X^2+2}$
\item $\frac{X^8-2X^7+X^6+X^2-4X+3}{X^7 - X^6}$
\item $\frac{X^5+16X^4+64X^3-X^2-16X-64}{X^4+17X^3+80X^2+64X}$
  (al final, utilizar $X - 1 = 2X - {(X+1)}$)
\end{enumerate}

\section{Soluciones de los ejercicios}

\subsection*{Ejercicio 1}

\begin{enumerate}
\item ${(X+17)}^2$

\item 
  ${(aY + b)}{(cY + d)} = acY^2 + {(ad+bc)}Y + bd = 6 Y^2 + Y -35$.
  Tratamos encontrar $a,b,c,d$ enteros tales que
  $ac = 6$, $ad+bc = 1$ y $bd= -35$. Podemos suponer $c \geq 0$ (sino, podemos
  cambiar el signo de todos los coeficientes y no cambia las egualidades).
  Por la primera relación, $c \in \{ 1, 2, 3, 6 \}$. Podemos también suponer
  que ${|c|} \leq {|a|}$ (sino, podemos intercambiar los dos factores). Entonces
  $c = 1$ (y $a=6$) o $c = 2$ (y $a=3$). Para cada caso, tratamos las
  posibilidades por $bd = -35 = -1 \times 5 \times 7$ y
  $ad+bc = 1$. Encontramos $c = 2$, $a=3$, $b=-7$ y $d=5$.
  Por consequencia, $P = {(2x+5)}{(3x-7)}$.

\item $P(0) = -289$, $P(-1) = -1 -33 -255 -289 = -512$,
  $P(1) = 1 + 33 + 255 - 289 = 0$. Tratamos escribir
  $P = {(X - 1)}{(X^2 + bX + c)} = X^3 + {(b - 1)}X^2 + {(c - b)}X - c$.
  Podemos tomar $c = 289$ y $b = 34$. Entonces
  $P = {(X-1)}{(X^2 + 34X + 289)} = {(X-1)}{(X+17)}^2$.
\end{enumerate}

\subsection*{Ejercico 2}

\begin{enumerate}
\item $Q = Z^2 + \left(\frac{5}{2} - \sqrt{3}\right) Z + \frac{37}{16} - \frac{5 \sqrt{3}}{4}$
\item Encontramos $k = Q - P = \frac{20\sqrt{3} + 37}{16}$.
\item Debemos resolver $3a^2 + b^2 = 37$ y $2ab=20$. Tratamos dividores de 20
  y finalmente $a = 2$ y $b = 5$ conviene.
\item $P = Q - k = \left( Z + \frac{5}{4} - \frac{\sqrt{3}}{2} \right)^2 -
  \left(\frac{2 \sqrt{3} + 5}{4} \right)^2$. Utilizamos la ley de los
  binomios conjugados:

  $$P = \left( Z + \frac{5}{4} - \frac{\sqrt{3}}{2} 
  - \frac{2 \sqrt{3} + 5}{4}
  \right) \left( Z + \frac{5}{4} - \frac{\sqrt{3}}{2} 
  + \frac{2 \sqrt{3} + 5}{4}
  \right) =
  \left( Z - \sqrt{3} \right) \left( Z + \frac{5}{2} \right)$$
\end{enumerate}

\subsection*{Ejercicio 3}

\begin{enumerate}
\item $Q = X^3-1, R = X + 4$
\item $Q = X - 1, R = X^2 -4X + 3$
\item $Q = X - 1, R = X^3+15X^2+48X - 64$
\end{enumerate}

\subsection*{Ejercicio 4}

\begin{enumerate}
\item Por el precedente ejercicio, se escribe $X^3 - 1 + \frac{X+4}{X^2+2}$.
\item Por el precedente ejercicio, se escribe
  $X-1 + \frac{X^2-4X+3}{X^7-X^6}$. En la fracción,
  $1$ es un cero trivial del numerador y del numerador entonces $X-1$ es un
  factor común y finalmente obtenemos
  $X-1 + \frac{1}{X^5} - \frac{3}{X^6}$
\item Por el precedente ejercicio, se escribe
  $X - 1 + \frac{X^3+15X^2+48X-64}{X^4+17X^3+80X^2+64X}$. En la fracción, $1$ es
  un cero trivial del numerador y $0, -1$ ceros triviales del denominador.
  Factorizando cada uno, nos damos cuenta de que $(X+8)^2$ es un factor común
  y finalmente obtenemos $X - 1 + \frac{X-1}{X{(X+1)}} = 
  X - 1 + \frac{2}{X+1} - \frac{1}{X}$.
\end{enumerate}
