\chapter{Números (1º Bimestre)}

\section{Sistemas de numeração}

\subsection*{Positional notation}

In this kind of system, we are given a base $b \geq 2$ and $b$ different symbols
$s_0, s_1, \ldots s_{b-1}$. Any nonnegative integer $n$ can be uniquely written
as a sum
%%
$$
n = a_0 + {a_1 \times b} + {a_2 \times b^2} + \ldots
{a_k \times b^k}
$$
%%
for some natural numbers $k$ and integers $0 \leq a_0, a_1, \ldots a_k \leq b-1$
with $a_k \neq 0$. This number $n$ is then written in positional notation
with the succession of symbols
%%
$$s_{a_k}s_{a_{k-1}} \cdots s_{a_1}s_{a_0}$$

Moroever, if we use an additional symbol $S$ as a decimal separator then
for any integers $l > 0$, the fraction $\frac{n}{b^l}$ is written as
%%
$$s_{a_k}s_{a_{k-1}} \cdots s_{l} S s_{l-1} \cdots s_{a_1}s_{a_0}$$

As an example, for the decimal base $b=10$ we use the
Hindu-Arabic digits $s_0 = 0, s_1 = 1, \ldots s_9 = 9$.
The decimal separator $S$
varies from country to country. In Brazil it is generally a comma ``$,$'' while
you often see a dot ``$.$'' in calculator.

\subsection*{Exercício 1 (binary and hexadecimal)}

\begin{enumerate}

\item The binary base $b=2$ uses two symbols $s_0 = 0$ and $s_1 = 0$.
  We let $N_0=77$ and $a_0 = 1$. Let $N_1 = \frac{N_0 - a_0}{2}$ and
  $a_1 = 0$ if $N_1$ is even and $a_1 = 1$ otherwise. What are the values
  of $N_1, a_1$?

\item We let $N_2 = \frac{N_1 - a_1}{2}$ and
  $a_2 = 0$ if $N_2$ is even and $a_2 = 1$ otherwise. What are the values
  of $N_2, a_2$?

\item Continue the same way to determine $N_3, a_3, N_4, a_4, N_3, a_4,
  N_4, a_5, N_5, a_5, N_6, a_6$. What is $N_7 = \frac{N_6-a_6}{2}$?

\item What can you say about
%%
  $${2^6 \times a_6} + {2^5 \times a_5} + {2^4 \times a_4} + {2^3 \times a_3}
  + {2^2 \times a_2} + {5 \times a_1} + a_0$$

\item What is the writting of $N_0 = 125$ in base $2$?

\item We consider $b=16$ and symbols $s_0=0,s_1=1,s_2=2,s_3=3,s_4=4,s_5=5,
  s_6=6,s_7=7,s_8=8,s_9=9,s_{10}=A,s_{11}=B,s_{12}=C,s_{13}=D,s_{14}=E,s_{15}=F$.
  Which numbers are written ``$4$'', ``$\text{D}$'' and ``$4\text{D}$'' in base
  $16$?

\item Write $4$ and $13$ in base $2$. What do you notice?

\end{enumerate}

\subsection*{Exercício 2 (Divisibility criterium, Babylonian numerals)}

\begin{enumerate}
\item What are the divisors of $b_1=10$?
\item For each of the divisors $d\neq1$, write the beginning of the list of
  multiple of $d$. What do you notice for the last digit?
\item What are the divisors of $b_2=16$?
  Indicate if the following numbers written in base $b=16$ (exercício 1)
  multiple of $4$: $\text{FA5BE7C29E3}$, $\text{E8EF3A2B9CD4}$,
  $\text{E8EFD43A2B9C}$, $\text{E8B9CE123D43E}$.
\item The oldest known positional notation is the Babylonian system which uses
  base $b_3=60$. What are the divisors of $b_3$? Which comment
  can you do on that?
\end{enumerate}

\subsection*{Sistemas de numeração na Antiguidade}

In antiquity, many numeral systems (e.g. Egyptian, Attic, Roman) are based
on this simple idea: use $k$ symbols $s_0,s_1,\dots, s_k$, each one representing
a value $1 = v_0 < v_1 < v_2 < \dots < v_k$. Writing a group of several
symbols among $s_0, s_1, \dots, s_k$ represents the number corresponding
to the sum of each value $v_0, v_1, \dots v_k$. For example, any number $N$
can be written with $N$ symbols $v_0$ but this takes a lot of place when
$N$ is large. To make the writing unique and optimal, one should use the
Euclidean division: take as many symbols $s_k$ as possible, then add as many
symbols $s_k$ as possible... and finish with $s_0$ symbols.

\subsection*{Exercício 3}

The Egyptian system used $7$ symbols: a stroke to represent the value $1$,
a heel bone to represent value $10$, a coil of rope to represent value
$100$, a Lotus to represent value $1000$, a finger to represent value $10000$,
an animal to represent value $100000$ and a man to represent value $1000000$.

\begin{enumerate}
\item How many symbols do you need to write $1205314$? $9999999$? $100000000$?
\item We consider $17$ ($7$ strokes and one heel bone) and
  $123$ ($1$ coil of rope, $2$ heel bones and $3$ strokes). How do you perform
  the sum of the two numbers? Is there a simple rule to compute the product?
\item Compare with the decimal positional system.
\end{enumerate}

\subsection*{Exercício 4}

The Roman numerals used the symbols $\text{I},\text{V},\text{X},\text{L},\text{C},\text{D},\text{M}$ with respective values $1,5,10,50,100,500,1000$. The
symbols are generally written from left to right in decreasing values. For
example $XVI$ represents $20+5+1=26$.

\begin{enumerate}
\item Write $1653$ with the roman numeral. Which number does
  $\text{MCCXXXV}$ represent?
\item Roman numerals introduced the following shorthand when a
  symbol follows a smaller one then this represents the substraction of the
  smaller from the larger. For example $9$ is generally written $\text{IX}$
  ($10-1$) instead of $\text{VIIII}$. How would you write $954$?
\item Special decorations (e.g. overbar) are used to
  indicate that a symbol is multiplied by $1000$. How does it improve over
  the Egyptian system?
\end{enumerate}

\section{Números negativos}

\subsection*{Introduction}

The reader should already be familiar with negative numbers,
for example these numbers are used to measure
negative temperatures in Celcius like $-5°\text{C}$. We have the classical
order: $-25°\text{C}$ is less than $-10°\text{C}$ which is itself less than
$5°\text{C}$.

Let's consider the sum of numbers.
Suppose that two people have $200\text{R\$}$ and $750\text{R\$}$ on
their respective bank accounts and decide to put their money together.
Then together they have $200+750=950\text{R\$}$ available. If the
two people have a debt of $200\text{R\$}$ and $750\text{R\$}$ then together
they have a debt of $950\text{R\$}$. If one has $750\text{R\$}$
on his bank account and the second a debt of $200\text{R\$}$, then the
total money available is $750 - 200 = 550\text{R\$}$. If one has a debt
of $750\text{R\$}$ and the second $200\text{R\$}$ then together they have debt
of $750-250=550\text{R\$}$.
On a bank account, the money available is written on your bank statement
with positive or negative numbers. For example if you have fifty reales on your
bank account then the bank statement contains will be written $50\text{R\$}$
while a debt of ten reales will be written $-10\text{R\$}$. When two people
put their money together we are doing the sum of the two amounts. As a
consequence, the previous discussion translates to the following definition of
sum: $750+200=950$, ${(-750)} + {(-200)} = -950$,
${(750)} + {(-200)} = 550$, ${(-750)} + {(200)} = -550$.

Now, let's consider the difference. Suppose one has $75\text{R\$}$ and that he
withdraws $50\text{R\$}$ to buy something, it remains $75-50=25\text{R\$}$
which is the classical definition of substraction. If he withdraw
$50\text{R\$}$ again, he now has a debt of $25\text{R\$}$ so with the
negative number notation it's $25 - 50 = -25\text{R\$}$. If the bank decides
to cancel a debt of $10\text{R\$}$ then it only remains a debt of
$15\text{R\$}$ and so with the negative number notation it's
${(-25)} - {(-10)} = -15$. If the bank decides to cancel another debt of
$20\text{R\$}$ then there is finally $5\text{R\$}$ on the bank account and
so with the negative number notation it's ${(-15)} - {(-20)} = 5$.

The same reasoning gives a natural defition of a product of a negative and a
positive number. The product of two negative numbers does not really have a
natural interpretation, but the desired properties of addition, substraction
and multiplication does not leave any choice, as shown by the following
exercise:

\subsection*{Exercício 5 (multiplication)}

Consider two numbers $x,y > 0$

\begin{enumerate}

\item Suppose that you have a debt of $1\text{R\$}$ on your bank account.
  What is the amount displayed on your bank statement?
  If the debt is multiplied by $x$, what is the amount displayed
  on your bank statement? How do your write this operation as a
  product of a positive and a negative number?
\item Multiplication of nonnegative numbers is commutative and associative,
  for example $2 \times 3 = 6 = 3 \times 2$ and
  ${(2 \times 3)} \times 4 = {6 \times 4} = 24 = {2 \times 12} =
  2 \times {(3 \times 4)}$. Assuming this is also true for negative
  numbers, show that
  ${(-x)} \times y = x \times {(-y)} = {(-1)} \times {(x \times y)}$
  and
  ${(-x)} \times {(-y)} =
  \left({(-1)} \times {(-1)}\right) \times \left(x \times y\right)$.
\item Suppose you
  have $1\text{R\$}$ on your bank account and that a debt of $1\text{R\$}$
  is added. What is the new amount on your bank statement?
  Write this statement as a sum of a positive and a negative number.
  What is the value of $1 + {(1 + (-1))} \times {(-1)}$?
\item For nonnegative numbers, multiplication is distributive over
  the addition, for example ${(1+2)} \times 3 =
  {1 \times 3} + {2 \times 3}$. Assuming this is
  also true for negative numbers, show that
  $1 + {(1 + (-1))} \times {(-1)} = {{(-1)} \times {(-1)}}$.
\item
  Conclude that
  ${(-x)} \times y = x \times {(-y)} = -{(x \times y)}$ and
  ${(-x)} \times {(-y)} = x \times y$
\end{enumerate}

\subsection*{Definition and operations}

For any positive number $x > 0$ we define a negative number denoted by
a minus sign in front of the number $x$, for example $-2$ or $-13.7$.
Given nonnegative numbers $0 \leq x \leq y$
(so $y - x$, $x + y$ and $x \times y$ are also nonnegative)
we define the following:

\begin{enumerate}
\item ${(-x)} + {(-y)} = {(-y)} + {(-x)} = -{(x+y)}$
\item $x + {(-y)} = {(-y)} + x = -{(y - x)}$
\item ${(-x)} + y = y + {(-x)} = {y - x}$
\item ${(-x)} - {(-y)} = {y - x}$
\item $x - {(-y)} = x + y$
\item ${(-x)} - y = -{(x + y)}$
\item ${(-x)} \times {(-y)} = {(-y)} \times {(-x)} = x \times y$
\item ${x \times {(-y)}} = {{(-y)} \times x} =
  {(-x)} \times y = y \times {(-x)} = -{x \times y}$
\end{enumerate}

For example $0 \leq 2 \leq 5$ so $-5 \leq -2 \leq 0$,
$2 + {(-5)} = -3$, $-2-5 = -7$, $-2 \times 5 = -10$ etc
This extends addition, multiplication and substraction to any number
(positive or not). By defition, addition and multiplication are still
commutative. We assume that this also preserves associativity:
${(A + B)} + C = A + {(B+C)}$ and
${(A \times B)} \times C = A \times {(B \times C)}$ for any numbers $A,B,C$.
We assume that this also preserves distributivity of the multiplication over
the addition and substraction:
$A \times {(B + C)} =  A \times B + A \times C$ and
$A \times {(B - C)} =  A \times B - A \times C$.

Finally, if $A \neq 0$ is a decimal number (maybe negative) and $N$ is a
nonnegative integer, we define the power
%%
$$A^N =
\underset{N\,\text{times}}{\underbrace{A \times A \times \ldots \times A}}$$
%%
with the special case $A^0 = 1$.

\subsection*{Exercício 6}

Calculate:

\begin{enumerate}
\item $2 + 7\times2 - 8$
\item $17-33$
\item $-8+11$
\item $2 - {(-8)}$
\item ${(-20)} - 11$
\item $-5 \times 8$
\item $-11 \times -7$
\item $7 - 11\times2 - \left(2-17+5\right) \times 3 + 5$
\item $\left( 2 - 8 + {(-17)} - {(-3)} \right) \times
  \left( {(-5)} + 11 - 7 \right) - 30$
\item $2.7 - 3.8 \times 4.1 - {(-9.3)}$
\item ${(-2)}^3$
\end{enumerate}

\section{Números racionais}

\subsection*{Definition and operations}

For any $p, q$ such that $p \geq 0$ and $q > 0$ we already
defined the fraction $\frac{p}{q} = p \div q$
in the 5ª série do Ensino Fundamental de Matemática e suas tecnologias and
showed that any nonnegative decimal numbers can be written as such a fraction
(for example $2.374 = \frac{2374}{1000}$). We also defined some operations
(for example $\frac{1}{2} + \frac{1}{4} = \frac{3}{4}$).
Thanks to the previous section,
we can now generalize this by allowing $p,q$ to be negative:

$$\frac{p_1}{q_1} + \frac{p_2}{q_2} = \frac{p_1 \times q_2 + q_1 \times p_2}{q_1 \times q_2}$$
$$\frac{p_1}{q_1} - \frac{p_2}{q_2} = \frac{p_1 \times q_2 - q_1 \times p_2}{q_1 \times q_2}$$
$$\frac{p_1}{q_1} \times \frac{p_2}{q_2} = \frac{p_1 \times p_2}{q_1 \times q_2}$$
$$\frac{p_1}{q_1} \div \frac{p_2}{q_2} = \frac{p_1 \times q_2}{q_1 \times p_2}$$

Finally, we for any numbers integers $p,q \neq 0$ and $N$ a nonnegative
number we define:

$$\left(\frac{p}{q}\right)^N =
\underset{N\,\text{times}}{\underbrace{\frac{p}{q} \times \frac{p}{q} \times
    \ldots \times \frac{p}{q}}} = \frac{p^N}{q^N}$$
$$\left(\frac{p}{q}\right)^{-N} =
\left(\frac{q}{p}\right)^{N} = \frac{q^N}{p^N}$$

with the special case $\left(\frac{p}{q}\right)^{0} = 1$.

\subsection*{Exercício 7}

Calculate:

\begin{enumerate}
\item $-\frac{3}{7} + \left(-\frac{2}{5}\right)$
\item $\frac{1}{6} - \frac{3}{8}$
\item $\frac{7}{6} \times \frac{-1}{5} \times \frac{-9}{-2}$
\item $\frac{-3}{4} \div \frac{5}{-16}$
\item $\left(\frac{2}{-3}\right)^{-3}$
\item $-{(0.9)}^2 \times 5 +
  \left( \frac{-17}{3} + \frac{2^2}{-6} \right) \div \frac{-140}{-21}$
\end{enumerate}

\subsection*{Comparison}

Suppose that $p_1,q_1,p_2,q_2 \geq 0$. Then
in the 5ª série do Ensino Fundamental de Matemática e suas tecnologias
we defined $\frac{p_1}{q_1}, \frac{p_2}{q_2} \geq 0$ and

$$\frac{p_1}{q_1} \leq \frac{p_2}{q_2} \Leftrightarrow
p_1 \times q_2 \leq p_2 \times q_1$$

We extend this ordering by noting that
$\frac{p_1}{q_1} = \frac{-p_1}{-q_1}$ and defining
$\frac{-p_2}{q_2} = \frac{p_2}{-q_2} \leq
\frac{-p_1}{q_1} = \frac{p_1}{-q_1} \leq 0 $
if $0 \leq \frac{p_1}{q_1} \leq \frac{p_2}{q_2}$.

\subsection*{Exercício 8}

Put the following rational numbers in increasing order:
$0, \frac{2}{3}, \frac{-1}{2}, \frac{3}{-4}, \frac{-7}{-8}, -\frac{3}{5}$

\section{Solução dos exercícios}

\subsection*{Exercício 1 (binary and hexadecimal)}

\begin{enumerate}

\item $N_1 = \frac{77-1}{2} = 38$ is even so $a_1=0$

\item $N_2 = \frac{38 - 0}{2} = 19$ is odd so $a_2=1$.

\item $N_3 = \frac{19-1}{2} = 9$ is odd so $a_3=1$.
  $N_4 = \frac{9-1}{2} = 4$ is even so $a_4=0$.
  $N_5 = \frac{4-0}{2} = 2$ is even so $a_5=0$.
  $N_6 = \frac{2-0}{2} = 1$ is odd so $a_6=1$. Finally,
  $N_7 = \frac{1-1}{2} = 0$.

\item We get
%%
  $${2^6 \times 1} + {2^5 \times 0} + {2^4 \times 0} + {2^3 \times 1}
  + {2^2 \times 1} + {2 \times 0} + 1 = 64+8+4+1=77=N_0$$

  Note that ``by construction'' we have
  $N_0 = 2N_1+a_0 = 2 \left(2N_2+a_1\right) + a_0 =
  2 \left(2\left(2N_3+a_2\right)+a_1\right) + a_0 = \dots =
  2 \left(2\left(2\left(2\left(2\left(2\left(2N_7+a_6\right)+a_5\right)+a_4\right)+a_3\right)+a_2\right)+a_1\right) + a_0 $ and $N_7=0$.

\item We have $s_{a_0} = s_{1} = 1 = a_0$ and similarly
  $s_{a_1} = a_1 = 0$, $s_{a_2} = a_2 = 1$, $s_{a_3} = a_3 = 1$,
  $s_{a_4} = a_4 = 0$, $s_{a_5} = a_5 = 0$ and $s_{a_6} = a_6 = 1$. By the
  previous equality, the writing of $N_0= 77$ in base 2 is then
  $1001101$.

\item $s_4=4$ represents $4$ and  $s_{13}=\text{D}$ represents $13$.
  ${s_4s_{13}}=\text{4D}$ represents
  $4 \times 16 + 13 = 77$.

\item We have $4=2^2$ and
  $13=12+1 = 2^2 \times 3 + 1 = 2^3 + 2^2 + 1$. As a consequence in base $2$,
  $4$ is written $100$ and $13$ is written $1101$. We notice that the
  concatenation of the digits of $4$ and $13$ in base $16$ (``4D'') and the
  concatenation of digits in base $2$ (``1001101'') represents the same
  number $N_0=77$.

\end{enumerate}

\subsection*{Exercício 2 (Divisibility criterium, Babylonian numerals)}

\begin{enumerate}
\item $1,2,5,10$
\item $2,4,6,8,10,12,14,16,18\ldots$ ($d=2$, always ends by $0,2,4,6,8$) ;
  $5,10,15,20,25,30\ldots$ ($d=5$, always ends by $0,5$) ;
  $10,20,30,40,50,60\ldots$ ($d=10$, always ends by $0$).
  We notice that the last digit of the multiple of $d$ belongs to a fixed set.
  Conversely, if the last digit is in this fixed set, then we have a multiple
  of $d$. Indeed, this is because we can always write
  $N = 10 \times N_1 + c_0$ where $c_0$ is the number of unity and
  $10 \times N_1$ is always multiple of $d$.

\item The divisors of $b_2=16$ are $1,2,4,8,16$. Hence
  to know if a number is multiple of $4$ we only need to check if the
  last digit is one of $0,4,8,C$.
\item The divisors of $60$ are $1,2,3,4,5,6,10,15,12,20,30,60$.
  Hence we get a quick way to determine if a number written in base $60$ is a
  multiple of these divisors (by checking the last digit). Actually, one can
  check that any multiple of $3,4,5$ is also a multiple of $60$ so this is the
  smallest base possible.
\end{enumerate}

\subsection*{Exercício 3}

\begin{enumerate}
\item For $1205314$, we need
  $4$ strokes, $1$ heel bone, $3$ coils of rope, $5$ Lotus,
  $2$ animals and $1$ man, that is $4+1+3+5+2+1=16$ symbols.
  Similarly, for $9999999$ we need $9+9+9+9+9+9+9=63$ symbols.
  For $100000000$ we will need to draw $100$ men!
\item For the sum, we just group the symbols:
  $7+3=10$ strokes, $1+2=3$ heel bones and $1+0=1$ coil of rope. We can
  also convert the $10$ strokes into one heel bone and so finally the sum
  is written with $1$ coil of rope and $4$ heel bones.
  The product is $2091$ that is $2$ Lotus, $9$ coil of ropes and $1$ stroke but
  there is not an obvious rule to deduce that easily.
\item The decimal positional system has a more compact writing and makes
  performing product easier.
\end{enumerate}

\subsection*{Exercício 4}

\begin{enumerate}
\item $\text{MVILIII}$, $1235$.
\item $\text{CMLIV}$
\item This allows to write large numbers like $100000000$ without too many
  symbols.
\end{enumerate}

\subsection*{Exercício 5 (multiplication)}

Consider two numbers $x,y > 0$

\begin{enumerate}

\item The amount initially displayed on the bank statement is $-1\text{R\$}$.
  It then becomes $-x\text{R\$}$. This can be written
  as ${{(-1)} \times x} = -x$.
\item By the previous question,
  ${(-x)} \times y = {({(-1)} \times x)} \times y$,
  $x \times {(-y)} = x \times {({(-1)} \times y)}$ and
  ${(-x)} \times {(-y)} =
  {({{(-1)} \times x)} \times {({(-1)} \times y})}$ by the previous questions.
  Using associativity and commutativity, this can be rewritten
  ${(-x)} \times y = x \times {(-y)} = {(-1)} \times {(x \times y)}$ and
  ${(-x)} \times {(-y)} =
  \left({(-1)} \times {(-1)}\right) \times \left(x \times y\right)$.
\item The available money and debt cancels and the
  new amount becomes $0\text{R\$}$. This can be written
  $1+{(-1)}=0$. Consequently,
  $1 + {(1 + (-1))} \times {(-1)} = 1 + 0 \times {(-1)} = 1$.
\item By distributivity and associativity,
  $1 + {(1 + (-1))} \times {(-1)} =
  1 + {1\times{(-1)}} + {{(-1)} \times {(-1)}} =
  {(1 + {(-1)})} + {{(-1)} \times {(-1)}} = {{(-1)} \times {(-1)}}$.
\item
  By second question, we have
  ${(-x)} \times y = x \times {(-y)} = {(-1)} \times {(x \times y)}$ and
  by the first question this is equal to $-{(x \times y)}$.
  By the second question, we have
  ${(-x)} \times {(-y)} = \left({(-1)} \times {(-1)}\right) \times \left(x \times y\right)$ and by the two previous questions, this is equal to ${x \times y}$.
\end{enumerate}

\subsection*{Exercício 6}

\begin{enumerate}
\item $8$
\item $-16$
\item $3$
\item $10$
\item $-31$
\item $-40$
\item $77$
\item $-10$
\item $-3.58$
\item $-8$
\end{enumerate}

\subsection*{Exercício 7}

\begin{enumerate}
\item $-\frac{29}{35}$
\item $-\frac{5}{24}$
\item $-\frac{21}{20}$
\item $\frac{12}{5}$
\item $-\frac{27}{8}$
\item $-5$
\end{enumerate}

\subsection*{Exercício 8}

\begin{enumerate}

\item We have $3 \times 5 < 3 \times 4$ so $\frac{3}{5} < \frac{3}{4}$ and
$-\frac{3}{5} > \frac{3}{-4}$.

\item We have $3 \times 2 = 6 > 5 = 1 \times 5$ so
  $\frac{3}{5} > \frac{1}{2}$ and
  $-\frac{3}{5} < \frac{-1}{2}$.

\item From the signs, we have $\frac{-1}{2} < 0 < \frac{2}{3}$.

\item We have $2 \times 8 = 16 < 3 \times 7 = 21$ so
  $\frac{2}{3} < \frac{7}{8} = \frac{-7}{-8}$

\end{enumerate}

Finally, $$\frac{3}{-4} < -\frac{3}{5} < \frac{-1}{2} < 0 < \frac{2}{3} < \frac{-7}{-8}$$
