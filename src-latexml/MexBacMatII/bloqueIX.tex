\chapter{Aplicas la estadística elemental}

\section{Población y muestra}

\subsection{Definición}

Una población estadística es cualquier conjunto de elementos sobre el que
realizamos observaciones. Una muestra estadística es un subconjunto de una
población que es representativo de la totalidad de la población.

\subsection{Ejercicio 1}

En 2012, la población de México era de 120 milliones de habitantes. El número
de habitantes inscritos para las elecciones federales eran de 80 milliones.
Un sondeo fue realizado en Junio 2012 con 12480 personas y 25\% de estas
personas indicarón una intención de voto para Enrique Peña Nieto.
Indique la población estadística correspondiente a la votación y la muestra
estadística utilizada para estimar el resultado.

\section{Medidas de tendencia central}

\subsection{Definición (data no agrupados)}

Consideramos $n$ datos $x_1, x_2, \ldots, x_n$. La media (aritmética) de esos
valores es
%%
$$\mu = E(x) =
\overline{x} = \frac{x_1 + x_2 + \ldots + x_n}{n} = \frac{1}{n}\sum_{i=1}^n x_i$$

La moda es el valor lo más repetido en estos datos. Es posible que no sea
única.

Si ordenados las valores
de menor a mayor, la mediana es el valor de la variable a la mitad de los datos
(si hay un número impar de datos)
o la media de las valores dos variables a la mitad (si hay un número par de
datos).

\subsection{Ejemplo}

Consideramos los datos $x_1 = x_4 = x_5 = 3$, $x_2 = x_6 = 1$,
$x_3 = 0$ y $x_7 = 10$. La media es $\frac{3+1+0+3+3+1+10}{7} = 3$. La moda
es $3$, valor tomado por $x_1, x_4, x_5$. Podemos ordenar las variables:
$x_3 = 0, 1, 1, {\color{red}3}, 3, 3, 10 = x_7$ y entonces la mediana es también $3$.

Ahora, añadimos el dato $x_8 = 1$. La media se vuelve
$\frac{3+1+0+3+3+1+10+1}{8} = 2.75$. Hay dos modas de frecuencia $3$ posibles:
$3$ (valor tomado por $x_1, x_4, x_5$) y
$1$ (valor tomado por $x_2, x_6, x_8$). Podemos ordenar las variables:
$x_3 = 0, 1, 1, {\color{red} 1}, {\color{red} 3}, 3, 3, 10 = x_7$ y entonces
la mediana es $\frac{1+3}{2} = 2$.

\subsection{Ejercicio 2}

Calcule la media, moda y mediana de estos datos

\begin{enumerate}
\item $1.60\text{m}, 1.58\text{m}, 1.66\text{m}, 1.72\text{m}, 1.67\text{m}, 1.73\text{m}, 1.65\text{m}, 1.66\text{m}, 1.70\text{m}, 1.80\text{m}$
\item $18 \text{años}, 25 \text{años}, 22 \text{años},  \text{22} \text{años},
   \text{22} \text{años},  \text{18} \text{años},
   \text{24} \text{años},  \text{20} \text{años},  \text{29} \text{años}$
\end{enumerate}

\subsection{Definición (data agrupados)}

En esta sección, consideramos datos agrupados en clases
${[a_1, b_1]}, {[a_2, b_2]}, \ldots, {[a_n, b_n]}$ con
$b_1 < a_2, b_2 < a_3 \ldots b_{n-1} < a_n$. Para $1 \leq k \leq n$, la
frecuencia de la clase $[a_k, b_k]$ es $f_k > 0$ y el total es 
$N = \sum_{k=1}^n f_k$. La media (aritmética) de estos datos es
%%
$$
\mu = E(x) =
\overline{x} = \frac{1}{N} \sum_{k=1}^n f_k \frac{a_k + b_k}{2}
$$

Para determinar la moda, consideramos el número $k$ tal que la frecuencia $f_k$ 
es máximal y ponemos
%%
$$
\text{Moda}
= a_k + \left( \frac{f_{k} - f_{k-1}}{{(f_k - f_{k+1})} + {(f_{k} - f_{k-1})}}
\right) \left(b_k - a_k \right)
$$

Aquí, suponemos que $1 < k < n$ y que $f_{k-1}, f_{k+1} < f_k$. Para utilizar
la formula de manera más general, podemos poner $f_0 = f_{n+1} = 0$ y
tomar $M = \frac{a_k+b_k}{2}$ si $f_k = f_{k-1} = f_{k+1}$. Igual al caso
no agrupados, la moda no es definida de manera única si hay muchos
intervalos de frecuencia máximal.

Para definir la mediana, consideramos las frecuencias acumuladas
$F_k = \sum_{i=1}^k f_i$ para $0 \leq k \leq n$ ($F_0 = 0$ y
$F_n = N$). Sea $k$ el menor
entero tal que $F_k \geq \frac{N}{2}$. La mediana es dado por la formula
%%
$$
\text{Mediana} =
a_k + \left( \frac{\frac{N}{2} - F_{k-1}}{f_k}
\right) \left(b_k - a_k \right)
$$

\subsection{Ejercicio 3}

Determine las clases, frecuencias y frecuencias acumuladas de estos datos.
Calcule la media, moda y mediana.

\begin{center}
\begin{tabular}{| c | c | c |}
  \hline
  Población (2013) & País o dependencia \\
  \hline
  100 a 200 millones & Brasil, México \\
  \hline
  30 a 100 millones &  Colombia, Argentina, Perú \\
  \hline
  10 a 30 millones & Venezuela, Chile, Ecuador, Guatemala,
  Cuba, Haití, Bolivia \\
  \hline
  1 a 10 millones & República Dominicana, Honduras, Paraguay, El Salvador,
  Nicaragua, Costa Rica, Rico Puerto Rico, Panamá, Uruguay \\
  \hline
  Menos de 1 millión & Guadalupe, Martinica, Guayana Francesa, San Martín,
  San Bartolomé, San Pedro y Miquelón \\
  \hline
\end{tabular}

Fuente: es.wikipedia.org
\end{center}

\subsection{Ejercicio 4}

Consideramos datos que podemos agrupar en frecuencias $f_1, \ldots, f_n$:
$x_1 = x_2 = \ldots = x_{F_1} < x_{F_1+1} = x_{F_1+2} = \ldots = x_{F_2} < \ldots
= x_{F_{n-1}} < x_{F_{n-1}+1} = x_{F_{n-1}+2} = \ldots =
x_{F_{n}}$ donde las $F_k$ son las frecuencias acumuladas.
Exprese la media y moda de esos datos no agrupados.

Si consideramos los clases con uno elemento
$a_k = b_k = x_{F_{k-1}} = x_{F_{k-1}+2} = \ldots = x_{F_{k}}$, exprese
la media y moda de esos datos agrupados y compare con las formulas para
datos no agrupados. ¿Que decir de la mediana?

\section{Medidas de dispersión}

\subsection{Definición (data no agrupados)}

Consideramos $n$ datos $x_1, x_2, \ldots, x_n$. Para obtener el rango, restamos el
valor mínimo del valor máximo. La varianza es
%%
$$
V(x) =
\frac{1}{n}
\sum_{i=1}^n \left(x_i - \overline{x}\right)^2
$$
%%
y la desviación típica
%%
$$
\sigma = \sqrt{V(x)}
$$

\subsection{Ejemplo}

Consideramos $x_1 = 0, x_2 = 20$ y $x_3 = 10$. La media es
$\frac{0+20+10}{3} = 10$. El rango es $20 - 0 = 20$. La varianza es
$\frac{\left(0-10\right)^2 + \left(10-10\right)^2 + \left(20-10\right)^2}{3}
= \frac{200}{3}$ y la desviación típica
$10 \sqrt{\frac{2}{3}} \approx 8.16$.

Ahora, añadimos los dato $x_4 = 13, x_5 = 17$. La media se vuelve
$\frac{0+20+10+13+17}{5} = 12$. El rango siempre es $20 - 0 = 20$ pero
la varianza es 
$\frac{\left(0-12\right)^2 + \left(10-12\right)^2 + \left(20-12\right)^2
+ \left(13-12\right)^2 + \left(17-12\right)^2}{5}
= \frac{238}{5}$ y la desviación típica
$\sqrt{\frac{238}{5}} \approx 6.89$.

\subsection{Ejercicio 5}

Calcule el rango, la varianza y la desviación típica de los datos del
ejercicio 2.

\subsection{Ejercicio 6}

Muestre que 
%%
$$
V(x) =
\frac{1}{n}
\left(\sum_{i=1}^n x_i^2\right) - \overline{x}^2
$$

\subsection{Definición (data agrupados)}

En esta sección, consideramos datos agrupados en clases
${[a_1, b_1]}, {[a_2, b_2]}, \ldots, {[a_n, b_n]}$ con
$b_1 < a_2, b_2 < a_3 \ldots b_{n-1} < a_n$. Para $1 \leq k \leq n$, la
frecuencia de la clase $[a_k, b_k]$ es $f_k > 0$ y el total es 
$N = \sum_{k=1}^n f_k$.

El rango es $b_n - a_1$. La varianza es
%%
$$
V(x) =
\frac{1}{N}
\sum_{k=1}^n f_k \left(\frac{a_k+b_k}{2} - \overline{x}\right)^2
$$
%%
y la desviación típica queda $\sigma = \sqrt{V(x)}$. 

\subsection{Ejercicio 7}

Calcule el rango, la varianza y la desviación típica de los datos del
ejercicio 3.

\section{Soluciones de los ejercicios}

\subsection{Ejercicio 1}

La población estadística correspondiente a la votación es el conjunto de los
habitantes inscritos para las elecciones federales. La muestra estadística
es el conjunto de los 12480 personas que fueron interrogadas.

\subsection{Ejercicio 2}

\begin{enumerate}
\item
La media es
$\frac{1.6+1.68+1.66+1.72+1.67+1.73+1.65+1.66+1.7+1.8}{10} = 1.687\text{m}$,
la moda $1.66\text{m}$ (dos veces) y la mediana
$\frac{1.68+1.67}{2} = 1.675\text{m}$.

\item
La media es $\frac{18+25+22+22+22+24+24+21+20}{9} = 22 \text{años}$.
La moda es $22 \text{años}$ (tres veces) y la mediana es tambíen $22 \text{años}$.

\end{enumerate}

\subsection{Ejercicio 3}

\begin{center}
\begin{tabular}{| c | c | c | c | }
  \hline
  $[a_k, b_k]$ & $f_k$ & $F_k$ \\
  \hline
  $[0,1]$ & $6$ & $6$\\
  \hline
  $[1,10]$ & $9$ & $15$ \\
  \hline
  $[10,30]$ & $7$ & $22$ \\
  \hline
  $[30,100]$ & $3$ & $25$ \\
  \hline
  $[100,200]$ & $2$ & $N=27$ \\
  \hline
\end{tabular}
\end{center}

La media es $\mu = \frac{1}{27} \left(
150 \times 2 + 65 \times 3 + 20 \times 7 + 5.5 \times 9 + 0.5 \times 6 \right)
\approx 25.5 \text{milliones}$. La moda se obtene para el interval $[1,10]$
($f_k=9$):

$$
1 + \frac{9-6}{9-6 + 9-7} \left(10-1\right) = 6.4 \text{milliones}
$$

Tenemos $\frac{27}{2} = 13.5$ entonces la mediana se obtene también para 
el interval $[1,10]$. Obtenemos:

$$
1 + \frac{13.5 - 6}{15} \left(10-1\right) = 5.5 \text{milliones}
$$

\subsection{Ejercicio 4}

La formula de la media es
$\overline{x} =
\frac{1}{N}{\left(x_1 + x_2 + \ldots + x_N\right)} =
\frac{1}{N} {\sum_{k=1}^n \sum_{i=1}^{f_k}  x_{(F_{k-1} + i)}}
$ y entonces
%%
$$
\overline{x} =
\frac{1}{N} {\sum_{k=1}^n f_k a_k}
$$

La moda es el valor $a_k = b_k$ con la frecuencia $f_k$ máximal.

Para datos agrupados, obtenemos $\frac{a_k+b_k}{2} = a_k$ y
$b_k - a_k = 0$. Entonces encuentramos las mismas formulas para la media y la
moda. 

De la misma manera, la formula de la mediana para los datos agrupados se vuelve
a $a_{k'}$ para un $1 \leq k' \leq n$. Cuando $N$ es par, es posible que
$F_{k-1} < \frac{N}{2} < F_{k-1}+1$ y la formula de
los datos no grupados es $\frac{b_{k-1}+a_{k}}{2}$ que no coresponde a
un $a_{k'}$. Pero podemos verificar que en otros casos, las formulas son 
son equivalentes para datos grupados y no grupados.

\subsection{Ejercicio 5}

\begin{enumerate}
\item
El rango es $1.80 - 1.58 = 0.22 \text{m} = 22 \text{cm}$.
La varianza es
%%
$$\frac{\left(1.6-1.675\right)^2+\left(1.68-1.675\right)^2+\left(1.66-1.675\right)^2+\left(1.72-1.675\right)^2+\left(1.67-1.675\right)^2+\left(1.73-1.675\right)^2+\left(1.65-1.675\right)^2+\left(1.66-1.675\right)^2+\left(1.7-1.675\right)^2+\left(1.8-1.675\right)^2}{10} =
0.002805$$
%%
 y entonces la desviación típica es
$\sqrt{0.002805} \approx 0.05 \text{m} = 5 \text{cm}$.

\item
El rango es $25-18 = 7 \text{ans}$. La varianza es
$\frac{38}{9} \approx  4.22$ y la desviación típica es
$\sqrt{38}{3} \approx 2 \text{ans}$.

\end{enumerate}


\subsection{Ejercicio 6}

$$
V(x) =
\frac{1}{n}
\sum_{i=1}^n \left(x_i - \overline{x}\right)^2
$$
%%
y $\left(x_i - \overline{x}\right)^2 =
x_i^2 - {2 x_i \overline{x}} + \overline{x}^2$ entonces
%%
$$
V(x) =
\frac{1}{n}
\left(\sum_{i=1}^n x_i^2\right)
- \frac{2}{n}
\left(\sum_{i=1}^n x_i\right) \overline{x}
+ \frac{n}{n} \overline{x}^2
$$
%%
y finalmente $V(x) = 
\frac{1}{n}
\left(\sum_{i=1}^n x_i^2\right) - 2\overline{x}^2 + \overline{x}^2 =
\frac{1}{n}
\left(\sum_{i=1}^n x_i^2\right) - \overline{x}^2$.


\subsection{Ejercicio 7}

El rango es $200 - 0 = 200 \text{milliones}$. La varianza es
%%
$$
V =
\frac{1}{27}
\left(
2 \times \left(150 - \mu \right)^2 +
3 \times \left(65 - \mu \right)^2 +
7 \times \left(20 - \mu \right)^2 +
9 \times \left(5.5 - \mu \right)^2 +
6 \times \left(0.5 - \mu \right)^2 
\right) \approx 1602
$$
%%
y la desviación típica $\sigma = \sqrt{V} \approx 40 \text{milliones}$.
