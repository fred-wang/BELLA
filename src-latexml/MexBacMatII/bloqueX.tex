\chapter{Empleas los conceptos elementales de probabilidad}

\section{Probabilidad clásica}

\subsection{Definición}

Wikipedia describe los conceptos elementales de probabilidad de la manera
siguiente.

La probabilidad es la característica de un evento, que hace que existan razones
para creer que éste se realizará.

La probabilidad $p$ de que suceda un evento $S$ de un total de $n$ casos
posibles igualmente probables es igual a la razón entre el número de
ocurrencias $h$ de dicho evento (casos favorables) y el número total de casos
posibles $n$.

$$
p=P\left(S\right)=\frac {h}{n}
$$

La probabilidad es un número (valor) que varia entre 0 y 1. Cuando el evento es
imposible se dice que su probabilidad es 0, si el evento es cierto y siempre
tiene que ocurrir su probabilidad es 1.

La probabilidad de no ocurrencia de un evento está dada por $q$, donde:

$$
q=P\left(no \; S\right)=1-\frac {h}{n}
$$

Sabemos que $p$ es la probabilidad de que ocurra un evento y $q$ es la
probabilidad de que no ocurra, entonces $p + q = 1$.

\subsection{Ejemplo}

Cuando lanzamos una moneda al aire, hay $n=2$ eventos posibles para el
lado de la moneda: ``aguila'' o ``sol''. Sea $S$ el evento de obtener ``sol''.
La probabilidad de obtener ``sol'' es $p = \frac{1}{2}$ y la probabilidad
de obtener ``aguila'' es $q = 1 - p = 1 - \frac{1}{2} = \frac{1}{2}$.
Aquí, suponemos que el evento que la moneda desaparece es imposible
es decir que su probabilidad es $0$.

\subsection{Ejercicio 1}

En una baraja de 52 naipes, cual es la probabilidad de tirar:

\begin{enumerate}
\item El siete de diamantes.
\item Un rey.
\item Una carta de picas.
\item Una figura (As, rey, reina o paje).
\item Un número par.
\end{enumerate}

\section{Leyes aditiva y Multiplicativa}

\subsection{Resumen}

Si $A$ y $B$ son does eventos mutuamente excluyente, tenemos
%%
$$
P(A \text{o} B) = P(A) + P(B)
$$

Si $A$ y $B$ son no excluyentes, $P(A \text{y} B) \neq 0$ y
%%
$$
P(A \text{o} B) = P(A) + P(B) - P(A \text{y} B)
$$

Si $A$ y $B$ son independientes,
%%
$$
{P(A \text{y} B)} = {P(A)} \times {P(B)}
$$

Si $A$ y $B$ son dependientes, y $P\left(B/A\right)$ es la probabilidad
de $B$ sabiendo que $A$ ocure, tenemos
%%
$$
{P(A \text{y} B)} = {P(A)} \times {P\left(B/A\right)}
$$

\subsection{Ejemplo}

Consideramos tres bolas en una caja, de color amarilla, blanca y celeste.
Sea $A$ el evento de tirar la bola amarilla, $B$ el de tirar la bola blanca
y $C$ el de tirar la bola celeste.

Si tiramos un bola de la caja, la probabilidad de tirar la bola amarilla o
la bola blanca es
$P(A \text{o} B) = P(A) + P(B) = \frac{1}{3} + \frac{1}{3}$ por que
$A$ y $B$ son excluyentes. Si $\bar{A}$ es la probabilidad de no tirar
la bola amarilla, $\bar{A}$ y $C$ no son mutuamente excluyente porque
$C$ implica $\bar{A}$:
$P(\bar{A} \text{y} C) = P(C) = \frac{1}{3} \neq 0$. Entonces

$
P(\bar{A} \text{o} C) = P(\bar{A}) + P(C) - P(\bar{A} \text{y} C)
= \frac{2}{3} + \frac{1}{3} - \frac{1}{3} = \frac{2}{3}
$.

Suponemos que hacemos este experimento dos veces.
Sean $A_1, B_1, C_1$ los eventos del primer experimento y
$A_2, B_2, C_2$ los eventos del segundo. $A_1$ y $B_2$ son independientes y
entonces la probabilidad de tirar la bola amarilla en el primer experimento
y la bola blanca en el segundo experimento es
$P(A_1 \text{y} B_2) = {P(A_1)} \times {P(B_2)} = \frac{1}{3} \frac{1}{3} =
\frac{1}{9}$.

Finalmente, tiramos una bola (sean $A_1', B_1', C_1'$ los eventos
corespondientes) y sin colocarla en la caja, tiramos una segunda bola
(sean $A_2', B_2', C_2'$ los eventos corespondientes). La probabilad de
tirar una bola amarilla y después una bola blanca es
${P(A_1' \text{y} B_2')} = P(A_1') \times {P(B_2' / A_1')} =
\frac{1}{3} \times \frac{1}{2} = \frac{1}{6}$.

\subsection{Ejercicio 2}

Consideramos una baraja de 52 naipes:

\begin{enumerate}

\item Tiramos una carta y, sin colacalar, tiramos una segunda carta.
  Sea $A$ el evento ``tirar un rey como primera carta'' y $B$ el evento
  ``tirar una figura como segunda carta''. Determine $P(A \text{y} B)$.

\item Sea $C$ el evento ``tirar un rey'' y $D$ el evento
  ``tirar un número par''. Determine $P(C \text{o} D)$.

\item Tiramos tres cartas (colocamos cada carta después haber tirarla).
  Sea $E$ el evento ``tirar el siete de diamantes como primera carta'',
  $F$ el evento ``tirar un rey como segunda carta'',
  $G$ el evento ``tirar una figura como tercera carta''. Determine
  $P(E \text{y} F \text{y} G)$.

\item Sea $H$ el evento ``tirar una carta de picas'' y $I$ el evento
  ``tirar una figura''. Determine $P(H \text{y} I)$ y $P(H \text{o} I)$.

\end{enumerate}

\subsection{Ejercicio 3 (distribución binomial)}

Realizamos un experimento con una probabilidad $p$ de éxito.

\begin{enumerate}
  \item ¿Cual es la probabilidad de fracaso del experimento?
  \item Si realizamos el experimento dos veces, cual es la probabilidad
    de tener ¿dos éxitos? ¿dos fracasos? ¿un éxito y un fracaso?
  \item Realizamos el experimento tres veces. Cual es la probabilidad de
    obtener un total de dos éxitos si al fin del segundo experimento ya hemos
    obtenido ¿dos éxitos? ¿dos fracasos? ¿un éxito y un fracaso?
  \item Realizamos el experimento $n$ veces y consideramos $0 \leq m \leq n$.
    ¿Cual es la probabilidad de obtener $m$ éxitos si al fin del
    experimento $n-1$ ya hemos obtenido ¿$m-1$ éxitos? ¿$m$ éxitos? ¿y en 
    otro casos?
  \item Notamos $N! = 1 \times 2 \times \ldots \times N$ y
    $\binom{N}{K} = \frac{N!}{{K!}{(N-K)!}}$. Sea $A_m^n$ la probabilidad
    de obtener $m$ éxitos en $n$ experimentos. Nuestra conjetura es
    $$
    P(A_m^n) = \binom{n}{m} p^m \left(1-p\right)^{n-m}
    $$
   
    Verifique que la formula es correcta par $n = 1, 2$. Si es correcta para
    $n-1$ experimentos, muestre que es también correcta para $n$
    experimentos.
  \item Jugamos a ``aguila'' o ``sol'' $10$ veces. ¿Cual es la probabilidad
    de obtener $9$ éxitos?
  \item Consideramos el experimento siguiente: tiramos $1$ carta en una baraja
    y ganamos si la carta es de corazones. Indique las probabilidades $p$ y
    $1-p$ de éxito/fracaso. Si realizamos este experimento $5$ veces,
    ¿Cual es la probabilidad de obtener $3$ o más éxitos?
\end{enumerate}


\section{Soluciones de los ejercicios}

\subsection{Ejercicio 1}

Hay 52 eventos equiprobables corespondente a cada carta.

\begin{enumerate}
\item $\frac{1}{52}$ (un siete de diamantes)
\item $\frac{4}{52} = \frac{1}{13}$ (4 reyes)
\item $\frac{9 + 4}{52} = \frac{1}{4}$ (9 números de picas y 4 figuras de picas)
\item $\frac{4 \times 4}{52} = \frac{4}{13}$ (4 figuras de 4 palos)
\item $\frac{4 \times 5}{52} = \frac{5}{13}$ (números 2, 4, 6, 8 y 10 de 4 palos)
\end{enumerate}

\subsection{Ejercicio 2}

\begin{enumerate}

\item En ejercicio 1, hemos encontrado $P(A) = \frac{1}{13}$
  y $P(\text{tirar una figura}) = \frac{16}{52}$. Aquí,
  $P(B / A) = \frac{16 - 1}{52 - 1}$ porque hay una figura/carta de menos
  cuando $A$ ocurre. Entonces
  $P(A \text{y} B) = P(A) \times P(B / A) = 
  \frac{1}{13} \frac{15}{51} = \frac{5}{221}$.

\item En ejercicio 1, hemos encontrado $P(C) = \frac{1}{13}$ y
  $P(D) = \frac{5}{13}$. Los dos eventos son mutuamente excluyente entonces
  $P(C \text{o} D) = \frac{1}{13} + \frac{5}{13} = \frac{6}{13}$.

\item Los tres eventos son independientes, entonces
  $P(E \text{y} F \text{y} G) = {P(E)} \times {P(F)} \times {P(G)}$.
  En el ejercicio 1, hemos encontrado
  $P(E) = \frac{1}{52}$, $P(F) = \frac{1}{13}$ y $P(G) = \frac{4}{13}$.
  Finalmente, $P(E \text{y} F \text{y} G) = \frac{1}{2197}$.

\item En ejercicio 1, hemos encontrado
  $P(H) = \frac{1}{4}$ y $P(I) = \frac{4}{13}$.
  $H \text{y} I$ es el evento ``tirar una figura de picas''. Hay $4$ figuras
  de picas (as, rey, reina y paje) entonces
  $P(H \text{y} I) = \frac{4}{52}=\frac{1}{13} = P(H) P(I)$ (en particular
  $H$ y $I$ son independientes).
  $P(H \text{o} I) = P(H) + P(I) - P(H \text{y} I) =
  \frac{1}{4} + \frac{4}{13} - \frac{1}{13} = \frac{25}{52}$.
\end{enumerate}

\subsection{Ejercicio 3 (distribución binomial)}

\begin{enumerate}
  \item $1-p$
  \item La probabilidad de obtener dos éxitos es $p \times p = p^2$.
    La probabilidad de obtener dos fracasos es ${(1-p)} \times {(1-p)}
    = {(1-p)}^2$. La probabilida de obtener un éxito y un fracaso 
    es $1 - p^2 - {(1-p)}^2$. Es también la probabilidad del evento
    ``éxito seguido de un fracaso o fracaso seguido de un éxito'' es decir
    $2 p{(1-p)}$.
  \item Si ya hemos obtenido $2$ éxitos, necesitamos un fracaso al tercer 
    experimento, lo que ocure con una probabilidad de $1-p$.
    Si hemos obtenido $0$ éxito, es imposible obtener un total de $2$ éxitos
    con el último experimento.
    Si ya hemos obtenido $1$ éxito, necesitamos un éxito al tercer 
    experimento, lo que ocure con una probabilidad de $p$.
  \item De la misma manera, es imposible obtener $m$ éxitos excepto
    si ya hemos obtenido $m-1$ o $m$ éxitos. Eso ocurre respectivamente
    con probabilidad $p$ y $1-p$.

  \item 
    Para $n=1$, obtenemos
    $P(A_0^1) = \binom{1}{0} p^0 \left(1-p\right)^{1-0} = 1-p$
    $P(A_1^1) = \binom{1}{1} p^1 \left(1-p\right)^{1-1} = p$.
    Para $n=2$, obtenemos
    $P(A_0^2) = \binom{2}{0} p^0 \left(1-p\right)^{2-0} = \left(1-p\right)^2$
    $P(A_2^2) = \binom{2}{2} p^2 \left(1-p\right)^{2-2} = p^2$ y
    $P(A_1^2) = \binom{2}{1} p^1 \left(1-p\right)^{2-1} = 2 p \left(1-p\right)$.
    Por la pregunta precediente,
%%
    $$
    P(A_m^n) = P(A_{m-1}^{n-1}) P(A_m^n / A_{m-1}^{n-1}) +
    P(A_{m}^{n-1}) P(A_m^n / A_{m}^{n-1}) =
    p {P(A_{m-1}^{n-1})} + \left(1-p\right) {P(A_{m}^{n-1})}
    $$

    Si suponemos que nuestra conjetura es corecta para ${P(A_{m-1}^{n-1})}$
    y ${P(A_{m}^{n-1})}$, obtenemos
%%
    $$
    P(A_m^n) = \left( 
    \binom{n-1}{m-1} + \binom{n -1}{m}
    \right) p^m \left(1-p\right)^{n-m}
    $$

    Pero
    $\binom{n-1}{m-1} + \binom{n -1}{m}  = 
    \frac{\left(n-1\right)!}{\left(n-m\right)! m!} 
    \left(m + \left(n-m\right)\right) = \binom{n}{m}$

  \item La probabilidad para un juego es $p = \frac{1}{2}$.
    Entonces
    $P(A_{10}^{10}) = \binom{10}{9} \left(\frac{1}{2}\right)^{9} 
    \left(\frac{1}{2}\right)^{10-9} = \frac{10}{2^{10}} = \frac{5}{512}
    = 0.009765625$

  \item $p = \frac{1}{4}$ y $1-p = \frac{3}{4}$.
    La probabilidad de obtener $3$ o más éxitos es
    $P(A_{3}^{5}) + P(A_{4}^{5}) + P(A_{5}^{5}) =
    \frac{53}{512} = 0.103515625$.
\end{enumerate}
