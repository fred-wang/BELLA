\chapter{Números (1º Bimestre)}

\section{Números naturais}

\subsection*{Definição}

In order to count a set of objects, we use the familiar numbers
$0, 1, 2, 3, \dots$. These are called the natural numbers.
We recall the classical operations on these numbers, that you should already be
familiar with:

\begin{enumerate}

\item If we group two sets of objects together then the total number of objects
  is given by the sum $+$ of the number in each set. For example, if I have
  two balls in the left hand and three balls in the right hand, I hold
  $2+3=5$ balls.

\item If we remove a subset of objects from a given set, then we obtain the
  subtraction of the two corresponding numbers. For example, I remove seven
  books from a bag containing ten books, it remains $10-7 = 3$ books in the bag.

\item If we have a group of sets of identical size, then the union of these
  sets is given by the multiplication of the size of the group by the size
  of each set in the group. For example, if I have three boxes, each one
  containing ten Firefox OS phones then I have $10 \times 3 = 30$ phones.

\item If we split a group of objects into sets of identical size, then
  the size of each set is given by the division of the size of the group by
  the number of sets. For example, if I $100 \text{R\$}$ between $4$ people
  then each people get $100 \div  4 = 25 \text{R\$}$.

\end{enumerate}

\subsection*{Exercício 1}

Solve the following problems using operations on natural numbers:

\begin{enumerate}
\item Fernanda has one cat, two dogs and seven birds. How many animals does she
  have?
\item Gabriel has ten coins of $25\text{R\$}$. He bought something that costs
  $100\text{R\$}$. How much does he have before buying the object?
  And after?
\item A group of 24 kids are playing football in the street. Each team
  has 8 players. How many teams are there?
\end{enumerate}

\subsection*{Múltiplos e divisores}

Given a natural number $N$, we can consider the set of natural numbers obtained
by multiplying $N$ by each natural number $0, 1, 2, 3, 4, \ldots$. For example
if $N = 3$, we have

$3 \times 0 = 0$
$3 \times 1 = 3$
$3 \times 2 = 6$
$3 \times 3 = 9$
$3 \times 4 = 12$
\ldots

The element in this set are called the multiples of $N$. Conversely, we say
that a nonzero natural number $N$ is a divisor of $M$, if $M$ is a multiple
of $N$. For example for
$M=12$, we already saw that $N=3$ is a divisor. Actually, $3 \times 4 = 12$
also shows that $4$ is a divisor of $12$. One can verify that the complete
list of divisors of $12$ are $1,2,3,4,6,12$.
We note that $1$ is always a divisor
of $M$ (because $M = 1 \times M$) that all nonzero numbers are divisor of $0$
(because $M \times 0 = 0$). Moreover, if $M \neq 0$ and $N$ is a divisor of
$M$ then $M = P \times N$ for some nonzero natural number $P$, which is
actually is another divisor of $M$.

\subsection*{Exercício 2}

Give the list of the $15$ first multiples of the following numbers:

\begin{enumerate}
\item $N=2$
\item $N=3$
\item $N=5$
\item $N=9$
\item $N=10$
\item $N=11$
\end{enumerate}

In the cases $N=2,5,10$ what can you say about the last digit of a given
multiple? In the cases $N=3,9$ what can you say about the sums of the digits of
a given multiple?

Compare the list of multiples for $N=2,5,10$. Similarly compare the two list
of multiples for $N=3,9$. Which inclusion do you notice? If $A$ is a divisor of
$B$ and $B$ a divisor of $C$ which relation can you find for $A,C$?

Given a multiple number $M$, we can consider the alternate
sum and subtraction of each the digit. For example for $M=12345$ we calculate
$1-2+3-4+5$. What do you notice for the multiples previously listed?
Compute $11 \times 19$ and $11 \times 653084$ and determine the alternate
sum and subtraction of their digits.

\subsection*{Exercício 3}

Indicate the list of divisors of the following numbers (you use the
properties found in exercício 2 as a help).

\begin{enumerate}
\item $M=35$
\item $M=24$
\item $M=60$
\item $M=17$
\end{enumerate}

\subsection*{Números primos}

A natural number $M$ is prime if it has exactly two distinct divisors
$1$ and $M$. In particular, this excludes $M=0$ since we assumed divisors
to be nonzero. This also excludes $M=1$ since in that case the divisors
are not distinct. Any natural numbers $M \geq 2$ always have at least two
divisors $1$ and $M$. We saw in Exercício 3 that it can have more divisors
but that $M=17$ was prime.

\subsection*{Exercício 4 (Crivo de Eratóstenes)}

\begin{enumerate}
\item Write the list of natural numbers $2 \leq M \leq 50$ and strike out the
  multiples of $2$. What are the remaining numbers?
\item The smallest number after $2$ in the remaining numbers is $3$.
  Strike out the multiples of $3$ in this remaining list.
  What are the remaining numbers?
\item The smallest number after $3$ in the remaining numbers is $5$.
  Strike out the multiples of $5$ in this remaining list.
  What are the remaining numbers?
\item Continue like this until you can not strike out any more numbers.
\item What can you say about the remaining numbers?
\end{enumerate}

\subsection*{Potências}

If $N$ is a natural number, we define its square $N^2 = N \times N$ and
its cube $N^3 = N \times N \times N$. For example $5^2 = 5 \times 5 = 25$ and
$3^3 = 3 \times 3 \times 3 = 9 \times 3 = 27$. More generally, if $p \geq 1$ is
a natural number, we define
%%
$$N^p =
\underset{p\,\text{times}}{\underbrace{N \times N \times \ldots \times N}}$$

For example $2^5 = 2 \times 2 \times 2 \times 2 \times 2 = 4 \times 4 \times 2
= 16 \times 2 = 32$.
By convention, we also define $N^0 = 1$ for all natural number $N \neq 0$.

\subsection*{Exercício 5 (basic properties of potências)}

\begin{enumerate}
\item Determine $2^0$, $2^1$, $2^2$, $2^3$, $2^4$, \ldots, $2^{10}$. Compare com
$0^2$, $2^2$, $3^2$, $4^2$, \ldots, $10^2$.
\item Compare $2^3$ e $3^2$
\item Compare $2^{\left(2^3\right)}$ e ${\left(2^2\right)}^3$
\item Compare $12^2 + 5^2$, $13^2$ e $\left(12+5\right)^2$.
\item Expresse $7^2 \times 7^9$ como uma potência de $7$.
\item Expresse $\left(11^{2}\right)^4$ como uma potência de $11$.
\item Expresse $13^7 \times 2^7$ como uma potência de $26$
\end{enumerate}

\subsection*{Sistema decimal}

When we write a number as sequence of digits like $1234$ we mean
$ {1 \times 1000} + {2 \times 100} + {3 \times 10} + 4 =
{1 \times 10^3} + {2 \times 10^2} + {3 \times 10} + {4 \times 10^0}$.
In general any natural number can be written in the form
%%
$$N = {c_n \times 10^n} + \dots + {c_2 \times 10^2} + c_1 \times 10 + c_0$$
%%
where $c_n,c_{n-1},\dots c_2,c_1,c_0$ are the $n$ digits of $N$ from left to
right, taking values $0,1,2,3,4,5,6,7,8,9$.

\subsection*{Exercício 6}

Let $N={10!} = 2 \times 3 \times 4 \times 5 \times 6 \times 7
\times 8 \times 9 \times 10$.

\begin{enumerate}
\item We first write $N = \left(2 \times 5\right) \times
\left(3 \times 9\right) \times
\left(4 \times 7\right) \times
\left(6 \times 8\right) \times 10$. For each term in parenthesis, find
the smallest multiple of $10$ that is greater or equal than that term.
Deduce that $N \leq 45 \times 10^5$.

\item Then we write $N =
  \left(2 \times 5\right) \times
  \left(3 \times 7\right) \times
  \left(4 \times 6\right) \times
  \left(8 \times 9\right) \times 10$. For each term in parenthesis, find
  the largest multiple of $10$ that is less or equal than that term.
  Deduce that $N \geq 28 \times 10^5$.

\item Without calculating the exact value of $N$, indicate the number of
  digits of $N$ in decimal form. Let $c_0, c_1, \ldots$ be the digits
  corresponding to the power $10^0, 10^1, \ldots$.

\item Show that $N$ is a multiple of $100$ and deduce $c_0, c_1$.

\item When we multiply two numbers, how do we obtain the digit of unity?
  Using the fact that $c_2$ is the digit of unity of $N \div 100$
  deduce $c_2=8$.

\item Using the fact that $4$ is a divisor of $N \div 100$,
  show that $c_3$ is even (note that $100 = 4 \times 25$).

\item Using the fact that $8$ is a divisor of
  $N \div 100$, show that $2$ is a divisor of $c_4+\frac{c_3}{2}$
  (note that $1000=8\times125$ $100=8\times12+4$ and $10=8+2$).

\item We now admit that $c_5=6$. From the two first questions, what can you say
  about $c_6$?

\item Using the fact that $9$ is a divisor of
  $N$, show that $9$ is a divisor of $c_4+c_3-1$ (use the criteria
  of divisibility of 9 guessed in exercício 2).

\item For each even digit for $c_3$, verify if we can find a digit $c_3$
  such that $c_4+\frac{c_3}{2}$ is even and $c_4+c_3+8$ multiple of $9$.
  Deduce the two remaining possibilities for $N$.

\item Calculate $5235 \times 7$ and use the fact that $N \div 100$ is
  a multiple of $7$ to deduce the value of $N$.
\end{enumerate}

\subsection*{Exercício 7 (Maya numerals)}

Maya numerals use $B$ digits, which are collection of $0 \leq d \leq 4$ dots and
$0 \leq b \leq 3$ bars. Each dot counts for $1$ and each bar counts for $5$.
Zero ($b=d=0$) is actually represented by a shell. Digits
$c_0, c_1, \ldots, c_n$ are written vertically to represent a number
%%
$$N = c_0 + B \times c_1 + B^2 \times c_2 + \ldots + B^n \times c_n$$

\begin{enumerate}
\item Write $3, 9, 12, 16$ in Maya numerals.
\item Determine the number of digits $B$ in this system.
\item Compute $20^2$, $20\times11$ and write $633$ using the Maya system.
\item Write the result of
  $\text{\textbullet\textbullet\textbullet} +
  \text{\textbullet\textbullet\textbullet\textbullet}$
  and
  ${\underline{\underline{\underline{\text{\textbullet\textbullet}}}}} +
  {\underline{\underline{\text{\textbullet}}}}$
  using the Maya system.
\item Indicate a way to determine if a Maya numeral is a multiple
  of $2,4,5,10,20$ (think of the
  criterion of divisibility in decimal base found in exercício 2).
\end{enumerate}

\section{Frações}

\subsection*{Definição}

For any natural numbers $p$ and any natural number $q\neq0$, $p \div q$ is
not necessarily a natural number. However, we can still interpret is as
``$p$ objects divided into $q$ parts'' and denote it by the fraction
$\frac{p}{q}$. For example if we divide $p=1$ cake
into $q=4$ pieces, we get a quarter of cake ($\frac{1}{4}$). If we have a
segment of length $7\text{cm}$ that we divide into $3$ parts then the length
of each segments is three thirds of centimer ($\frac{7}{3} \text{cm}$).
$p$ is called the numerator and $q$ the denominator.

\subsection*{Exercício 8}

Place on a line the values $0,1,2,3,4,\frac{6}{2},\frac{1}{4},\frac{9}{4},
\frac{10}{3}$.

\subsection*{Equivalência de frações}

We consider a segment of length $p$ divided into $q$ parts. We then join
$k$ copies of this segment to form a segment of length $k \times p$. Each
part of length $\frac{p}{q}$ in the initial segment is also copied and we get
$k \times q$ parts each of size $\frac{k \times p}{k \times q}$. Hence we have
%%
$$
\frac{k \times p}{k \times q} = \frac{p}{q}
$$
%%
that is a fraction does not change when we multiply (or divide) the numerator
and denominator by a same nonzero natural number. In particular if we consider
the equality of two fractions
%%
$$
\frac{p_1}{q_1} = \frac{p_2}{q_2}
$$
%%
then multiplying the numerator and denominator of the left hand side fraction
by $q_2$ and the numerator and denominator of the right hand side by $q_1$ we
get
%%
$$
\frac{p_1 q_2}{q_1 q_2} = \frac{q_1p_2}{q_1q_2}
$$
%%
That is, two segments of length $p_1q_2$ and
$q_1p_2$ divided by the same number of parts $q_1q_2$ gives pieces of identical
lengths. This necessarily means that the initial segments had the same length
and so
%%
$$p_1q_2 = q_1p_2$$

More generally, if the pieces in the first segment are shorter than the
pieces in the second segment then that's also true for the length of the
initial segments. It also easy to see that the converse is true and finally
%%
$$
\frac{p_1}{q_1} \leq \frac{p_2}{q_2} \Leftrightarrow
p_1q_2 \leq q_1p_2
$$

If we consider a fraction $\frac{p}{q}$ and $k$ is a common divisor for
$p,q$ then we can simplify the fraction by dividing the numerator and denomitor
by $k$. Of course $p,q$ always have a common divisor $k=1$ we can consider
the largest of if ($1 \leq k \leq p,q$)
to simplify the fractions as much as possible. When this largest common divisor
is $k=1$, we can not simplify the fraction any further and the fraction is
said to be in reduced form.

As an example, $\frac{2}{3} \leq \frac{6}{8}$ because
$2 \times 8 = 16 \leq 18 = 6 \times 3$. And actually
$\frac{6}{8}$ can be reduced to $\frac{3}{4}$ (dividing the numerator and
denominator by $2$).

\subsection*{Exercício 9}

Compare the following fractions. Then write them in reduced form.

$$\frac{8}{4},\frac{49}{21},\frac{24}{12},\frac{35}{15},\frac{120}{105}$$

\subsection*{Operações}

Since we can interpret fractions $\frac{p_1}{q_1}$ and $\frac{p_2}{q_2}$ as
lengths of segments we can naturally define
$\frac{p_1}{q_1} + \frac{p_2}{q_2}$ as the sum of the lengths of these
segments. Similarly, if $\frac{p_1}{q_1} \leq \frac{p_2}{q_2}$ we can
define $\frac{p_2}{q_2} - \frac{p_1}{q_1}$ as the difference of the lengths.
Let's see that these are actually fractions. To do that, we multiply
the numerator and denominator in the first fraction by $q_2$ and
the numerator and denominator in the second fraction by $q_1$. We get
%%
$$
\frac{p_1}{q_1} = \frac{p_1 q_2}{q_1 q_2} = {(p_1 q_2)} \times \frac{1}{q_1 q_2}
$$
%%
$$
\frac{p_2}{q_2} = \frac{q_1 p_2}{q_1 q_2} = {(q_1 p_2)} \times \frac{1}{q_1 q_2}
$$

Now we can see the fractions as the length obtained by taking joining
${p_1 q_2}$ (respectively ${q_1 p_2}$) segments of length
$\frac{1}{q_1 q_2}$. And so we naturally get
%%
$$
\frac{p_1}{q_1} + \frac{p_2}{q_2} = \frac{{p_1q_2}+{q_1p_2}}{q_1q_2}
$$
%%
and if $\frac{p_1}{q_1} \leq \frac{p_2}{q_2}$,
%%
$$
\frac{p_2}{q_2} - \frac{p_1}{q_1} = \frac{{q_1p_2}-{p_1q_2}}{q_1q_2}
$$

Of course we can then reduce the fractions as much as possible and we can use
any common multiple of $q_1,q_2$ in the denominator. For example
$\frac{1}{2}+\frac{3}{4}-\frac{1}{3} =
\frac{6}{12}+\frac{9}{12}-\frac{4}{12}=\frac{6+9-4}{12} = \frac{11}{12}$

\subsection*{Exercício 10}

Compute the following fractions:

\begin{enumerate}
  \item $\frac{1}{2}+\frac{1}{4}+\frac{1}{8}+\frac{1}{16}$
  \item $\frac{11}{5}-\frac{7}{4}$
  \item $\frac{49}{21} - \frac{35}{15}$
  \item $\frac{9}{3}+\frac{2}{5}-\frac{1}{2}$
  \item $\frac{1}{5}+\frac{7}{9}-\frac{2}{3}+\frac{1}{45}$
\end{enumerate}

\section{Solução dos exercícios}

\subsection*{Exercício 1}

\begin{enumerate}
\item Fernanda has $1+2+7=10$ animals.
\item Gabriel initially has $10 \times 25 = 250\text{R\$}$. After buying the
  object, it remains $250 - 100 = 150\text{R\$}$.
\item There are $24 \div 8 =6$ teams.
\end{enumerate}

\subsection*{Exercício 2}

\begin{enumerate}
\item $ 0, 2, 4, 6, 8,10,12,14,16,18,20,22, 24,26,28$
\item $ 0, 3, 6, 9,12,15,18,21,24,27,30,33, 36,39,42$
\item $ 0, 5,10,15,20,25,30,35,40,45,50,55, 60,65,70$
\item $ 0, 9,18,27,36,45,54,63,72,81,90,99,108,117,126$
\item $  0, 10, 20, 30, 40, 50, 60, 70, 80, 90,100,110,120,130,140$
\item $  0, 11, 22, 33, 44, 55, 66, 77, 88, 99,110,121,132,143,154$
\end{enumerate}

We note that the last digit of multiple of $2$ is always $0,2,4,6,8$,
the last digit of multiple of $5$ is always $0,5$ and the last digit of the
multiple of $10$ is always $0$.

For $N=3$, the sums of digits is given by
$0,3,6,9,1+2=3,1+5=6,1+8=9,2+1=3,2+4=6,2+7=9,3+0=3,3+3=6,3+6=9,3+9=12,4+2=6$.
and are themselves (smaller) multiples of $N=3$. The same happens for $N=3$,
for example $1+2+6=9$ or $9+9=18$ are multiples of $9$.

We note that the list of multiples of $10$ is included in the list
of multiple of $5$ and in the list of multiple of $2$. Similarly the list of
multiples of $N=9$ is included in the list of multiples of $3$. Hence we see
that if $A$ is a divisor of $B$ and $B$ a divisor of $C$ then $A$ is also a
divisor of $C$.

For $N=11$, we have $1-1=2-2=3-3=\ldots=9-9=1+1-0=1-2+1=1-3+2=1-4+3=1-5+4=0$.
However, for the multiples $11\times19 = 190 + 19 =209$ and
$11 \times 653084 = 6530840 + 653084 = 7183924$ we
find $2-0+9=11$ and $7-1+8-3+9-2+4=22$ so the critera is rather that the
alternate sum and subtraction is a multiple of $11$.

\subsection*{Exercício 3}

\begin{enumerate}
\item $1,35$ are divisors of $35$. By the previous criterions,
  $35$ can not be divided by $2$ or $3$ and so neither by
  $2\times2=4$ or $2\times3=6$.
  However, it can be divided by $5$ and we note $35=5\times7$ and so
  $7$ is also a divisor of $35$. Any divisor strictly larger than $7$ would be
  paired with a divisor strictly smaller
  than $35 \div 7=5$ but we have already verified all these
  numbers. So the divisors of $35$ are $1,5,7,35$.
\item $1,24$ are divisors of $24$.
  By the previous criterions, $5$ is not a divisor
  of $24$ but $2,3$ are divisors of $M$ and we get the corresponding divisors
  $24\div2=12,24\div3=8$. Actually, we notice that $24=4\times6$ which provides
  two more divisors. But again, any divisor strictly larger
  than $6$ would be paired
  with a divisor strictly
  smaller than $4$, that we have already verified. So the
  divisors of $24$ are $1,2,3,4,6,8,12,24$.
\item $1,60$ are divisors of $60$. By the previous criterions, $9$ is not
  a divisor of $60$ but $2,3,5,10$ are. These are are paired with
  $60\div2=30$, $60\div3=20$, $60\div5=12$ and $60\div10=6$. We moreover
  notice that $60=4\times15$. Any divisor strictly larger than $6$ would be
  paired with a divisor strictly smaller than $10$. Hence it remains to
  verify that the products $7\times7=49, 7\times8=56, 8\times8=64$ are not
  valid decomposition of $60$. So the divisors of $60$ are
  $1,2,3,4,5,6,10,12,15,20,30,60$.
\item $1,17$ are divisors of $17$. By the previous criterons $2,3$ are not
  divisors of $17$ and a fortiori $4=2\times2$ is not a divisor either.
  Any divisor larger or equal to $5$ would be paired with a divisor
  smaller or equal to $17 \div 5 \leq 20 \div 5 = 4$ but we already have
  verified all possibilities.

\end{enumerate}

\subsection*{Exercício 4 (Crivo de Eratóstenes)}

\begin{enumerate}
\item The initial list is
  $2,3,4,5,6,7,8,9,10,11,12,13,14,15,16,17,18,19,20,21,22,23,24,25,26,27,28,29,30,31,32,33,34,35,36,37,38,39,40,41,42,43,44,45,46,47,48,49,50$
  the remaining numbers are $2$ and the odd ones:
  $2,3,5,7,9,11,13,15,17,19,21,23,25,27,29,31,33,35,37,39,41,43,45,47,49$.
\item $2,3,5,7,11,13,17,19,23,25,29,31,35,37,41,43,47,49$.
  These are $2,3$ and the numbers that are not multiple of $2$ or $3$.
\item $2,3,5,7,11,13,17,19,23,29,31,37,41,43,47,49$
  These are $2,3,5$ and the numbers that are not multiple of $2,3,5$.
\item
  We eliminate the multiples of $7$ larger than $7$ and get
  $2,3,5,7,11,13,17,19,23,29,31,37,41,43,47$.
  For $11$, the multiples to consider are $22,33,44$ but they already have
  been excluded. More generally, for $N \geq 13$, we have
  $4 \times N \geq 4 \times 13 = 52 > 50$ so the multiples to consider
  are $2N$ and $3N$ which have already be eliminated at the beginning.
\item By construction, the remaining numbers are not multiple of a number
  smaller than them in the remaining list (otherwise they would have been
  eliminated when we considered that number).
  Also, they are not a multiple of a struck out number smaller
  than them (otherwise they would have been eliminated at the same time as
  this struck out number). So the only possible divisors of a remaining
  numbers $M$ is $1$ and $M$, that is we obtain the list of prime numbers
  between $1$ and $50$: $2,3,5,7,11,13,17,19,23,29,31,37,41,43,47$.

\end{enumerate}

\subsection*{Exercício 5 (basic properties of potências)}

\begin{enumerate}
\item Obtemos
  $2^0 = 1$,
  $2^1 = 2$,
  $2^2 = 4$,
  $2^3 = 8$,
  $2^4 = 16$,
  $2^5 = 32$,
  $2^6 = 64$,
  $2^7 = 128$,
  $2^8 = 256$,
  $2^9 = 512$,
  $2^{10} = 1024$ e
  $0^2 = 0$,
  $1^2 = 1$,
  $2^2 = 2$,
  $3^2 = 9$,
  $4^2 = 8$,
  $5^2 = 25$,
  $6^2 = 36$,
  $7^2 = 49$,
  $8^2 = 64$,
  $9^2 = 81$,
  $10^2 = 100$.
  Notamos que $2^n$ é torna-se muito maior que $n^2$ a medida que $n$ cresce.
\item $2^3 = 8 \neq 9 = 3^2$
\item $2^{\left(2^3\right)} = 2^8 = 256 \neq 64 = 4^3 = {\left(2^2\right)}^3$
\item $12^2 + 5^2 = 144 + 25 = 169 = 13^2 \neq 17^2 = \left(12+5\right)^2$
\item $7^2 \times 7^9 = \left(7 \times 7\right)
  \times \left(7 \times 7 \times 7 \times 7 \times 7 \times 7 \times 7 \times 7 \times 7\right) = 7^{2+9} = 7^{11}$
\item $\left(11^{2}\right)^4 = \left(11 \times 11 \right)
  \times \left(11 \times 11 \right) \times \left(11 \times 11 \right) \times \left(11 \times 11 \right) = 11^{4 \times 2} = 11^8$
\item $13^7 \times 2^7 =
  \left(13 \times 13 \times 13  \times 13  \times 13  \times 13 \times 13 \right)
  \times
  \left(2 \times 2 \times 2  \times 2  \times 2  \times 2 \times 2 \right)$,

  $13^7 \times 2^7 =
  {\left(13 \times 2\right)
  \times  \left(13 \times 2\right)
  \times  \left(13 \times 2\right)
  \times  \left(13 \times 2\right)
  \times  \left(13 \times 2\right)
  \times  \left(13 \times 2\right)
  \times  \left(13 \times 2\right) } = 26^7$
\end{enumerate}

\subsection*{Exercício 6}

\begin{enumerate}
\item We get $N = \left(2 \times 5\right) \times
\left(3 \times 9\right) \times
\left(4 \times 7\right) \times
\left(6 \times 8\right) \times 10 \leq
10 \times 30 \times 30 \times 50 \times 10 \leq 45 \times 10^5$.

\item We get $N =
  \left(2 \times 5\right) \times
  \left(3 \times 7\right) \times
  \left(4 \times 6\right) \times
  \left(8 \times 9\right) \times 10 \geq 10 \times 20 \times 20 \times
  70 \geq 28 \times 10^5$.

\item The previous inequalities show that
  $2 \times 10^6 \leq N \leq 5 \times 10^6$ so $N$ has 6 digits
  and can be written
  ${c_6 \times 10^6} + {c_5 \times 10^5} + {c_4 \times 10^4} +
  {c_3 \times 10^3} + {c_2 \times 10^2} + {c_1 \times 10} + c_0$ with
  $2 \leq c_6 \leq 5$.

\item Indeed, $10 \times 2 \times 5 = 100$ is a divisor of $N$. So
  the writing of $N$ ends by two zeros that is $c_0 = c_1 = 0$.

\item The digit of unity in the product is obtained by multiplying the digit of
  unity in each number, for example $12 \times 13 = 156$ with
  $2 \times 3 = 6$. Here we want to determine the digit of unity of
  $3 \times 4 \times 6 \times 7 \times 8 \times 9$. We apply our remark
  several times:
  the digit of unity of $3 \times 4=12$ is $2$. Then the digit of unity of
  $2 \times 6 = 12$ is again $2$. Then the digit of unity of
  $2 \times 7 = 14$ is $4$. Then the digit of unity of
  $4 \times 8 = 32$ is $2$. Finally the digit of unity of
  $2 \times 9 = 18$ is $8$. So $c_2=8$.

\item We write

  $N \div 100 =
  {c_6 \times 10^4} + {c_5 \times 10^3} + {c_4 \times 10^2} +
  {c_3 \times 10} + 8 =
  4 \times \left( 25 \times \left(
  {c_6 \times 10^2} + {c_5 \times 10} + {c_4} \right) + 2 \right) +
  10c_3$. Since $4$ is a divisor of $N \div 100$ it is also a divisor
  of $10c_3$. This shows that $10c_3=4K$ for some natural $K$ and
  so $5c_3=2K$ for some natural $K$. In particular $5c_3$ is even, which is
  only possible if $c_3$ is even too.

\item By the same reasoning, we obtain that
  ${4 c_4} + {2c_3} = 8K$ for some natural $K$ and so
  $c_4 + \frac{c_3}{2}$ is even.

\item
  The first question shows that the two first digits of $N$ satisfy
  $28 \leq {10c_6 + c_5} \leq 45$. If we have the digit $c_5=6$, then
  the only possibility is $c_6=3$.

\item
  $9$ is a divisor of $N$ so
  it is also a divisor of
  $c_6+c_5+c_4+c_3+c_2+c_1+c_0 = 3+6+c_4+c_3+8+0+0 = 9 + c_4+c_3+8$
  and so a divisor of $c_4+c_3+8$. This means at least $c_4+c_3 \geq 1$ since
  and so $9$ is actually a divisor of $9 + (c_4+c_3-1)$ that is a divisor
  of $c_4+c_3-1$.

\item If $c_3=0$, $c_4-1$ is a multiple $9$ for $c_4=1$ which contradicts the
  fact that $1=c_4 + \frac{c_3}{2}$ is even.
  If $c_3=2$, $c_4+1$ is a multiple $9$ for $c_4=8$ which contradicts the fact
  that $9=c_4 + \frac{c_3}{2}$ is even.
  If $c_3=4$, $c_4+3$ is a multiple $9$ for $c_4=6$ which satisfies
  $8=c_4 + \frac{c_3}{2}$ even.
  If $c_3=6$, $c_4+5$ is a multiple $9$ for $c_4=4$ which contradicts the fact
  that $7=c_4 + \frac{c_3}{2}$ is even.
  If $c_3=8$, $c_4+7$ is a multiple $9$ for $c_4=2$ which satisfies
  $6=c_4 + \frac{c_3}{2}$ even.
  This gives two possibilities for $N$:
  $N = 3664800$ and $N=3628800$.

\item We get $5235 \times 7 = 36645$ so
  $5235 \times 7 < 36648 < 5236 \times 7$. Since $N \div 100$ is a multiple
  of seven, the only possibility is $N=3628800$. We can indeed verify
  that by a direct calculation.
\end{enumerate}

\subsection*{Exercício 7 (Maya numerals)}

\begin{enumerate}
\item $3=\text{\textbullet\textbullet\textbullet}$
  $9=5+4=\underline{\text{\textbullet\textbullet\textbullet\textbullet}}$
  $12=2\times5+2=
  \underline{\underline{\text{\textbullet\textbullet}}}$
  and $16=3\times5+1=
  \underline{\underline{\underline{\text{\textbullet}}}}$
\item Determine the number of digits $B$ in this system.
\item $20^2=400$, $20\times11=220$. So
  $633=1\times20^2+11\times20+13$ is written with three digits:
  $1=\text{\textbullet}$,
  $11=2\times5+1=\underline{\underline{\text{\textbullet}}}$ and
  $13=2\times5+3=\underline{\underline{\text{\textbullet\textbullet\textbullet}}}$.
\item To add two digits, we sum up the dots and bars.
  Five dots are converted into one bar and four bars are converted into
  one dot and shift to the following digit.
  So $\text{\textbullet\textbullet\textbullet} +
  \text{\textbullet\textbullet\textbullet\textbullet}$
  has $7$ dots that is one bar plus $2$ dots.
  ${\underline{\underline{\underline{\text{\textbullet\textbullet}}}}} +
  {\underline{\underline{\text{\textbullet}}}}$
  has $5$ bars and $3$ dots that is $1$ dot for the digit $c_1$ and
  $1$ bar plus $3$ dots for the digit $c_0$.
  We can verify that the first sum is $3+4={(1\times5+2)}=7$ and
  the second $17+11={(3\times5+2)} + {(2\times5+1)} =
  {20c_1+c_0}=
  {20\times1 + {(1\times5+3)}} = 28$.

\item
  In base $10$, it is enough to consider
  the last digit to determine whether a number is a multiple of a divisor
  of $10$. For example multiple of $10$ ends by a $0$, multiple of $5$ by
  $0,5$ and multiple of $2$ by $0,2,4,6,8$.
  Similarly, in the Maya system it is enough to consider the last digit to
  determine if a number is a multiple of a divisor $N=2,4,5,10,20$ of $B=20$.
  More precisely, multiple of $N=20$ ends by $0$ (a shell), multiple of $10$
  ends by $0,10=2\times5$ (a shell or 2 bars), multiple of
  $5$ ends by $0,5,10,15$ (a shell or one, two or three bars), multiple
  of $4$ ends by $0,4,8,12,16$ (a shell or the sum of bar and dots is $4$)
  of $2$ ends by $0,2,4,\ldots18$ (a shell of the sum of bar and dots is even).

\end{enumerate}

\subsection*{Exercício 8}

We have $\frac{6}{2} = 3$, $\frac{9}{4} = \frac{8+1}{4} = 2+\frac{1}{4}$
and $\frac{10}{3} = \frac{9+1}{3} = 3 + \frac{1}{3}$.

\begin{center}
\begin{tikzpicture}
  \draw (0,0) --(5,0);
  \draw (1,-.2) --(1,.2);
  \draw (2,-.1) --(2,.1);
  \draw (3,-.1) --(3,.1);
  \draw (4,-.1) --(4,.1);
  \draw (5,-.1) --(5,.1);
  \draw (0,-.2) --(0,.2);
  \path (0,.5) node (n0) {$0$};
  \path (1,.5) node (n1) {$1$};
  \path (2,.5) node (n2) {$2$};
  \path (3,.5) node (n3) {$\frac{6}{2}$};
  \path (4,.5) node (n4) {$4$};

  \draw (0.25,-.1) --(0.25,2);
  \draw (0.5,-.1) --(0.5,.1);
  \draw (0.75,-.1) --(0.75,.1);
  \path (0.25,2.2) node (n025) {$\frac{1}{4}$};

  \draw (2.25,-.1) --(2.25,2);
  \path (2.25,2.2) node (n94) {$\frac{9}{4}$};

  \draw (3.333333333333,-.1) --(3.333333333333,2);
  \draw (3.666666666666,-.1) --(3.666666666666,.1);
  \path (3.333333333333,2.2) node (n103) {$\frac{10}{3}$};



\end{tikzpicture}
\end{center}


\subsection*{Exercício 9}

We first compare the fraction:
\begin{enumerate}
\item $8 \times 21 = 160 + 8 < 200 - 4 = 4 \times 49$ so
$\frac{8}{4} < \frac{49}{21}$.
$8 \times 12 = 80+16=4 \times 24$ so $\frac{8}{4} = \frac{24}{12}$.
$8 \times 15 = 80+40 <  120+20 = 4 \times35$ so
$\frac{8}{4} < \frac{35}{15}$.
$8 \times 105 = 840 > 480 = 4 \times 120$ so
$\frac{8}{4} > \frac{120}{105}$.

\item $49 \times 15 = 490 + 245 = 735 = 700 + 35 \times 21$ so
$\frac{49}{21} = \frac{35}{15}$.
\end{enumerate}

Then we determine the reduced fractions:

\begin{enumerate}

\item
  $\frac{24}{12}=\frac{8}{4}=\frac{2}{1}=2$ (we divided by $4$)
\item
  $\frac{49}{21}=\frac{35}{15}=\frac{7}{3}$ (we divided by $5$)

\item
  $\frac{120}{105} = \frac{24\times5}{21\times5}=
  \frac{24}{21}=\frac{8\times3}{7\times3} = \frac{8}{3}$.
\end{enumerate}

Finally

$\frac{120}{105} < \frac{24}{12}=\frac{8}{4}=2 < \frac{49}{21} =
\frac{35}{15}=\frac{7}{3}$

\subsection*{Exercício 10}

\begin{enumerate}
  \item $\frac{15}{16}$
  \item $\frac{9}{20}$
  \item $0$
  \item $\frac{29}{10}$
  \item $\frac{1}{3}$
\end{enumerate}
