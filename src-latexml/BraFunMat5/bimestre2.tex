\chapter{Números/Relações (2º Bimestre)}

\section{Números decimais}

\subsection*{Representação}

In the first bimestre we described the decimal representation of a natural
number $N$ with digits $c_n,c_{n-1}, \ldots c_0$ (from left to right) as a sum
%%
$$N = {c_n \times 10^n} + \dots + {c_2 \times 10^2} + c_1 \times 10 + c_0$$

Here the ``weight'' of digit $c_0$ is $1$, the weight of $c_1$ is $10$,
and we continue like with successive power of $10$. We can generalize this
by considering digits $c_{-1}$ of weight $\frac{1}{10}$, $c_{-2}$ of weight
$\frac{1}{10^2}$ etc A decimal number is such a number of the form
%%
$$D = {c_n \times 10^n} + \dots + {c_2 \times 10^2} + c_1 \times 10 + c_0 +
{c_{-1} \times \frac{1}{10}} +
{c_{-2} \times \frac{1}{10^2}} + \cdots +
{c_{-m} \times \frac{1}{10^{m}}}
$$
%%
where $n,m \geq 0$ and the digits take values $0,1,2,3,4,5,6,7,8,9$. The
typical way to represent such a decimal number is to write the digits
$c_n,c_{n-1},\dots c_{1}, c_{0}, c_{-1}, c_{-2}, \dots c_{-m}$ from left to right
and to use a decimal separators between the digit $c_0$ and $c_{-1}$. This
separator varies from country to country.
In Brazil, we generally use a comma but
in computers or calculator we often use a point as a decimal mark. For example
%%
$${1234.567} = {1234,567} = 1000 + 200 + 30 + 4 + \frac{5}{10} + \frac{6}{100} +
\frac{7}{1000}$$

Decimal number should be familiar when using money: for example
$10.50 \text{R\$}$ means $10$ reales and $50$ centavos.
If $m=0$ that is there are no digits $c_{-1}, \dots$ then the decimal mark can
be ommited: we find our classical representation of natural numbers.

\subsection*{Exercício 1}

\begin{enumerate}
\item Order the following decimal numbers: $0.001$, $10$, $2.89$, $2.890$,
  $10.00$, $50.2013$.
\item Is the decimal representation of a number (natural or decimal) unique?
\item Are all the natural numbers some decimal numbers?
\item
  By definition, $12.34 = 10 + 2 + \frac{3}{10} = \frac{4}{100}$. Can you write
  it as a fraction? Are all the decimal numbers some fractions?
\item Given a decimal writing $D$ as above, what is the decimal
  writing of $10D$? of $100D$? of $D \div 10$? of $D \div 100$?
  We can consider $D=5$ or $D=0.7$.
\end{enumerate}

\subsection*{Exercício 2 (difficult)}

We now consider the fraction $f=\frac{1}{3}$ and decimal number
%%
$$D = {c_n \times 10^n} + \dots + {c_2 \times 10^2} + c_1 \times 10 + c_0 +
{c_{-1} \times \frac{1}{10}} +
{c_{-2} \times \frac{1}{10^2}} + \cdots +
{c_{-m} \times \frac{1}{10^{m}}}
$$
%%
where $n,m \geq 0$ and the digits take values $0,1,2,3,4,5,6,7,8,9$. We want
to know if we can write $f = D$.

\begin{enumerate}
\item Calculate $3D$ for $D=1000.0003$, $D=1.0001$, $D=0.4001$,
  $D=0.2$, $D=0.29$ and $D=0.299$ and compare with $3f = 1$.
\item If $c_0 \geq 1$, show that $3D > 1$. Can we have $f = D$?
\item Deduce that $f=D$ implies $c_n = c_{n-1} = \ldots = c_{0} = 0$. We now assume
  that all these digits are zero.
\item Show that $c_{-1} \geq 4$ implies $3D > 1$.
\item Now write
%%
  $$3D = \frac{3 c_{-1}}{10} +
  \frac{3 \left(c_{-2} + \frac{c_{-3}}{10} + \cdots \frac{c_{-m}}{10^{m-2}}\right)}{100}
  $$

  Show that the rightmost fraction can not exceed $\frac{3}{10}$. Deduce
  that if $c_{-1} \leq 2$ we have $3D < 1$.

\item Deduce that $f=D$ implies $c_{-1}=3$. We now assume that below.

\item Show that $10f - 3 = f$. What is the decimal writing of $10D - 3$?
  Deduce that $f=D$ implies $c_{-2} = 3$. We now assume that below.

\item Show that $100f - 33 = f$. What is the decimal writing of $100D - 33$?
  Deduce that $f=D$ implies $c_{-3} = 3$.

\item Deduce that $f=D$ implies that the decimal writing of
  $D$ is $0.333333 \cdots 333$. What about the decimal writing of $3D$?

\item Conclude that $f = \frac{1}{3}$ is not a decimal number.

\end{enumerate}

\subsection*{Transformação em fração decimal}

As we have seen in Exercício 1, we can always write a decimal number as a
fraction
%%
$$D=
\frac{{c_n \times 10^{n+m}} + \dots + {c_2 \times 10^{2+m}} + {c_1 \times 10^{1+m}} + {c_0 \times 10^m} +
{c_{-1} \times 10^{m-1}} +
{c_{-2} \times 10^{m-2}} + \cdots +
{c_{-m}}}{10^{m}}
$$
%%
that we can of course reduce. Said otherwise, the numerator is a natural number
with the same digits as $D$ and the denominator is a power of $10$ to shift
the decimal separator as wanted.
For example $679.203=\frac{679203}{1000}$. We can of course try to reduce the
fraction, as seen in bimestre 1.

Exercício 2 showed that fractions like $\frac{1}{3}$ can not always be
written as a decimal number. However, we can always try to approximate
a fraction using "an infinite number of digits after the decimal separator".
For example we would have $\frac{1}{3} = 0.3333333333333333 \dots$ with an
infinite number of $3$'s. We notice that the decimal writing is eventually
periodic that is we repeat the same digit $3$ at the end.

More generally, a fraction $\frac{a}{b}$ can be approximated by doing the usual
long division algoritm for $a \div b$ that we learn at school.
Because the remainder can only take a finite number of values
$0, 1, \dots {b-1}$ and so at some point we will just repeat the same operation
in the algorithm. As an example let's determine the decimal expansion of
$\frac{24}{11}$:

$$\frac{24}{11} = \frac{22+2}{11} =
2 + \frac{1}{10} \times \frac{20}{11}$$
$$\frac{20}{11} = \frac{11+9}{11}  = 1 + \frac{1}{10} \times \frac{90}{11}$$
$$\frac{90}{11} = \frac{88+2}{11}  = 8 + \frac{1}{10} \times \frac{20}{11}$$
$$\frac{20}{11} = \frac{11+9}{11}  = 1 + \frac{1}{10} \times \frac{90}{11}$$
$$\frac{90}{11} = \frac{88+2}{11}  = 8 + \frac{1}{10} \times \frac{20}{11}$$

We can continue like this with the sequence of remainers $9,2,9,2,9,\ldots$
repeating indefinitely and so $\frac{24}{11} = 2.181818181818\ldots$ with the
sequence $18$ repeating indefinitely.

\subsection*{Exercício 3}

Determine a fração reduzida correspondente to the following decimal numbers:

\begin{enumerate}
\item $a = 23$
\item $b = 12.35$
\item $c = 1.40625$
\end{enumerate}

\subsection*{Exercício 4}

Utilize o algoritmo da divisão para determinar a expansão decimal das frações:

\begin{enumerate}
\item $\frac{3}{8}$
\item $\frac{203}{16}$
\item $\frac{17}{13}$
\end{enumerate}

\subsection*{Operações}

Since we saw that all the decimal numbers are fractions, we can now perform
sum and subtraction of decimal numbers. However, these operations are really
just like the operations on natural numbers, except that we have to take care
of the decimal separators. For example
$12.3+1.5 = 12+\frac{3}{10}+1+\frac{5}{10} = 13+\frac{8}{10} = 13.8$ is
exactly the same as $123+15=138$.

\subsection*{Exercício 5}

Calculate

\begin{enumerate}
\item $1.23+0.16$
\item $121.230+3.57$
\item $103.739-2.12$
\item $2.2-1.35$
\item $1.29+3.22-2.3-1.2$
\end{enumerate}

\section{Sistemas de medida}

\subsection*{Medidas de comprimento, massa e capacidade}

The International System of Units provide some units of measure that should
be already been familiar:

\begin{enumerate}
\item The length or distance (meter denoted $\text{m}$)
\item The mass (kilogram denoted $\text{kg}$).
\item The time (second denoted $\text{s}$).
\item The temperature (kelvin denoted $\text{K}$).
\end{enumerate}

From these units, we can derive other units for surface ($\text{m}^2$),
volume ($\text{m}^2$), speed ($\text{m}/\text{s}$) or density
($\text{kg}/\text{m}^3$). People often use other equivalent units like
the litre ($1\text{L}=0.001 \text{m}^3$), the minutes/hours ($60$ and $3600$
seconds) or the degree Celsius and Fahrenheit for the temperature.

\subsection*{Exercício 6}

The density of liquid water is $1000 \text{kg}/\text{m}^3$ while a liter is
defined by $1\text{L}=0.001 \text{m}^3$. What is the mass of one liter of
water?

\subsection*{Exercício 7}

The meter was initially defined to be $\frac{1}{10000000}$ of the distance
between the of the distance from the Earth's equator to the North Pole.
It was then replaced by the length of the path travelled by light in vacuum
during a time interval of $\frac{1}{299792458}$ seconds.

\begin{enumerate}
\item What is the approximate speed of the light in $10^8 \text{m}/\text{s}$?
\item What is the approximate circumference of the earth in $10^7 \text{m}$?
  What can you say about electromagnetic communication on Earth?
\item The distance from the Earth to the Moon is approximately
  $4\times10^8 \text{m}$. How long does communication between the Earth and
  the Moon take (in seconds)?
\item The minimal and maximal distance between Earth to Mars are approximately
  $5\times10^{10} \text{m}$ and $4 \times10^{11} \text{m}$
  How long does communication between the Earth and the Mars take (in minutes)?
\end{enumerate}

\subsection*{Exercício 8}

If $T_1$ is a temperature in Kelvin and $T_2$ a temperature in degree Celsius
then we can go from one to another by the formula $T_2 = T_1 + 273.15$.

Guilherme who lives near Campos do Jordão says that the temperature in his
city is around $10\text{°C}$ in August. His friend Flavio from Rio de Janeiro
says that it's is about twice as hot where he lives.

\begin{enumerate}
\item What is the temperature in Kelvin at Campos do Jordão?
\item What would then be the temperature in Kelvin in Rio de Janeiro according
  to Flavio's remark?
\item What would then be the temperature in degree Celsius in Rio de Janeiro?
\item What can you say about Flavio's remark?
\end{enumerate}

\section*{Sistema métrico decimal: múltiplos e submúltiplos da unidade}

We saw that unity of mass in the International System of Units ``kilogram''.
The prefix ``kilo'' actually means $1000$ times a gram, that is
$1 \text{kg} = 1000 \text{g}$. This is generally convenient to express
large or small values instead of using power of tens as in previous exercises.
More generally, we have the following prefixes and factors:

\begin{center}
  \begin{tabular}{ l | l | l }
    \hline
tera& 	T& 	1000000000000 \\ \hline
giga& 	G& 	1000000000 \\ \hline
mega& 	M& 	1000000 \\ \hline
kilo& 	k& 	1000 \\ \hline
hecto& 	h& 	100 \\ \hline
deca& 	da& 	10 \\ \hline
deci& 	d& 	0.1 \\ \hline
centi& 	c& 	0.01 \\ \hline
milli& 	m& 	0.001 \\ \hline
micro& 	\textmu&0.000001 \\ \hline
nano& 	n& 	0.000000001 \\ \hline
pico& 	p& 	0.000000000001 \\
    \hline
  \end{tabular}
\end{center}

If we consider a square of side $1m$ then its surface is $1\text{m}^2$.
If we consider a square of side $1\text{dm} = 10\text{m}$ then its surface is
$1\text{dm}^2 = 10 \times 10 = 100 \text{m}^2$. We then note that the ``deci''
factor $10$ is squared, that is $1\text{dm}^2 = 100 \text{m}^2$. Similar
thing happens for volume in $\text{dm}^3$ etc. People often use alternative
units like Litres or Hectares so that the prefixes and factors keep the same
meaning.

\subsection*{Exercício 9}

Convert the following units:

\begin{enumerate}
  \item $10000 \text{m}$ in $\text{km}$
  \item $0.5 \text{L}$ in $\text{dL}$
  \item $1 \text{Å} = 10^{-10} \text{m}$ in $\text{nm}$
\end{enumerate}

\subsection*{Exercício 10}

\begin{enumerate}
\item Convert $5 \text{L}$ in $\text{m}^3$.
\item What is the volume in $1\text{m}^3$ of a cube of side $1\text{cm}$?
\item Convert $5 \text{cL}$ in $\text{cm}^3$.
\end{enumerate}

\subsection*{Exercício 11}

1 are represents a surface of 100 square metres, while 1 alqueire paulista is
2.42 hectares. Bruna owns a land of 5 alqueire paulista. What is the superficie
of her land in $\text{km}^2$?

\section{Solução dos exercícios}

\subsection*{Exercício 1}

\begin{enumerate}
\item $0.001 < 2.89 = 2.890 < 10 = 10.00 < 50.2013$.
\item No, because we can add some zero digits like $1=01=001=001.000$.
\item Yes, by definition since the case $c_{-1}=c_{-2}=\dots c_{-m} = 0$
  gives the writing of a natural number in base 10.
\item
  $12.34 = 10 + 2 + \frac{3}{10} = \frac{4}{100} =
  \frac{1000+200+30+4}{1000} = \frac{1234}{1000} =
  \frac{617}{500}$. In general, we can always put everything in a big fraction:

$$D=
\frac{{c_n \times 10^{n+m}} + \dots + {c_2 \times 10^{2+m}} + {c_1 \times 10^{1+m}} + {c_0 \times 10^m} +
{c_{-1} \times 10^{m-1}} +
{c_{-2} \times 10^{m-2}} + \cdots +
{c_{-m}}}{10^{m}}
$$

\item Just like multiplying a natural number by $10,100$ etc is like
  appending $0$ to the decimal writing,
  multiplying of dividing by a power of $10$ is like shifting the decimal
  separator and adding $0$'s if necessary.
  For example for $D=5$, we have
  $10D=50$, $100D=500$, $D \div 10 = 0.5$ and $D \div 100 = 0.05$.
  If $D=0.7$,
  $10D=7$, $100D=70$, $D \div 10 = 0.07$ and $D \div 100 = 0.007$.

\end{enumerate}

\subsection*{Exercício 2 (difficult)}

\begin{enumerate}
\item We get $3000.0009 > 1$, $3.0003 > 1$, $1.2003 > 1$,
  $0.6<1$, $0.87<1$ and $0.897 < 1$.
\item As seen in the previous question, if $c_0 \geq 1$
  we have $3D \geq 3c_0 \geq 3 > 1$ so we can not have $f=D$.
\item We also saw in the first question
  that $c_{3} \geq 1$ implies $3D \geq 3000 > 1$. More generally
  if we want $f=D$, all the digits before the decimal mark must be zero:
  $c_n = c_{n-1} = \ldots = c_{0} = 0$.
\item Again, this is seen in the first question.
  If $c_{-1} \geq 4$ we have
  $3D \geq 3c_{-1} \times \frac{1}{10} \geq \frac{12}{10} > 1$
\item
  As suggested in the first question,
  the best we can do for the rightmost part is to take
  $c_{-2} = c_{-3} = \dots c_{-m} = 9$. But the numerator in
%%
  $$
  \frac{3 \left(c_{-2} + \frac{c_{-3}}{10} + \cdots \frac{c_{-m}}{10^{m-2}}\right)}{100}
  $$
%%
  will never exceed $30$ and so the fraction will never exceed $\frac{3}{10}$.
  If $c_{-1} \leq 2$, $\frac{3 c_{-1}}{10} \leq \frac{6}{10}$ so
  the sum can not exceed $3D=\frac{6+3}{10}=\frac{9}{10}<1$.

\item If $f=D$ then we have neither $c_{-1} \leq 2$ nor $c_{-1} \geq 4$ by
  the two previous questions. So the only possibility is $c_{-1}=3$.

\item $10f-3=10 \frac{1}{3} - 3 = \frac{10-9}{3} = \frac{1}{3}=f$
  Moreover, by the previous assumptions
  $10D - 3$ will have $0$ digits for dizains, centains etc
  $c_{-1} - 3 = 0$ for the unity and
  $c_{-2}, c_{-3}, \dots c_{-m}$ for the digits after the decimal mark.
  But then with $f=10f-3=10D - 3$ we come back to the case of the third
  question, except that the digits have been shift. In particular the role
  of $c_{-2}$ is now the role of $c_{-1}$ in previous questions and
  we deduce as above that $c_{-2} = 3$.

\item
  $100f-33=10 \frac{100}{3} - 33 = \frac{100-99}{3} = \frac{1}{3}=f$
  The decimal expansion of $100D - 33$
  will again be 0 for the digits before the decimal separator and
  $c_{-3}, c_{-4}, \dots c_{-m}$ for the digits after the decimal mark.
  With the same reasoning, we get $f=D$ implies $c_{-3} = 3$.

\item We just continue the reasoning until $c_{-m}$. Then
  $D = 0.333333 \cdots 333$ and
  $3D = 0.999999 \cdots 999 < 1$

\item If $D=f$ we have shown above that $3D < 1$ and
  so $D < \frac{1}{3} = f$. This is a contradiction and so the assumption
  $D=f$ is impossible. Hence $f = \frac{1}{3}$ can not be a decimal number.

\end{enumerate}

\subsection*{Exercício 3}

\begin{enumerate}
\item $a = \frac{23}{1}$
\item $b = \frac{1235}{100}=\frac{247}{20}$
\item $c = \frac{140625}{100000} = \frac{45}{32}$
\end{enumerate}

\subsection*{Exercício 4}

\begin{enumerate}
\item $0.375$
\item $12.6875$
\item $1.3076923076923076923\ldots$ (not a decimal number)
\end{enumerate}

\subsection*{Exercício 5}

\begin{enumerate}
\item $1.39$
\item $124.8$
\item $101.619$
\item $0.85$
\item $1.01$
\end{enumerate}

\subsection*{Exercício 6}

$0.001 \times 1000 = 1 \text{kg}$

\subsection*{Exercício 7}

The meter was initially defined to be $\frac{1}{10000000}$ of the distance
between the of the distance from the Earth's equator to the North Pole.
It was then replaced by the length of the path travelled by light in vacuum
during a time interval of $\frac{1}{299792458}$ seconds.

\begin{enumerate}

\item The speed of light is
  $299792458 \approx 3 \times 10^8 \text{m}/\text{s}$.

\item The circumference of the earth is $4$ times
  the distance from the Earth's equator to the North Pole that is
  $4 \times 10000000 = 4 \times 10^7 \text{m}$.

  The electromagnetic communication on Earth should
  is essentially of the same magnitude of speed as the light in vacuum.
  ($10^8 \text{m}/\text{s}$) distances on Earth are of the same magnitude
  as the circumference ($10^7 \text{m}$) so communication should be
  instantaneous.

\item
  $\frac{4\times10^8}{3 \times 10^8} \approx 1.33 \text{s}$

\item
  At least
  $\frac{5\times10^{10}}{3 \times 10^8} \approx 166 \text{s} \approx 3
  \text{minutes}$ and at most
  $\frac{4\times10^{11}}{3 \times 10^8} \approx 1333 \text{s} \approx 
  22 \text{minutes}$
\end{enumerate}

\subsection*{Exercício 8}

If $T_1$ is a temperature in Kelvin and $T_2$ a temperature in degree Celsius
then we can go from one to another by the formula $T_1 = T_2 + 273.15$.

Guilherme who lives near Campos do Jordão says that the temperature in his
city is around $10\text{°C}$ in August. His friend Flavio from Rio de Janeiro
says that it's is about twice as hot where he lives.

\begin{enumerate}
\item $10+273.15=283.15\text{K}$.
\item $2 \times 283.15 = 566.3\text{K}$
\item $566.3 - 273.15 = 293.15\text{°C}$!
\item Flavio's remark is not correct. He is probably using a relative unit
  (degree Celcius) instead of an absolute unit (Kelvin) to measure the
  temperature.
\end{enumerate}

\subsection*{Exercício 9}

\begin{enumerate}
  \item $10000 \text{m} = 10\text{km}$
  \item $0.5 \text{L}  = 5 \text{dL}$
  \item $1 \text{Å} = 10^{-10} \text{m} = 0.1 \text{nm}$
\end{enumerate}

\subsection*{Exercício 10}

\begin{enumerate}

\item $5 \text{L} = 0.005 \text{m}^3$.

\item The side is $1\text{cm} = 0.01 \text{m}$ so the volume is
  $0.01 \times 0.01 \times 0.01 = 0.000001 \text{m}^3 = 1 \text{cm}^3$.

\item $5 \text{cL} = 0.05 \text{L} = 0.00005 \text{m}^3 = 50 \text{cm}^3$.

\end{enumerate}

\subsection*{Exercício 11}

$50 \times 2.42 = 121$ hectares, that is $121 \times 100 = 12100$ ares,
$12100 \times 100 = 1210000 \text{m}^2$. Finally, we get
$1210000 \div 1000^2 = 1.21 \text{km}^2$.
