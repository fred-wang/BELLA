\chapter{Números (1º Bimestre)}

\section{Números Racionais}

\subsection*{Definição}

Se $p$ e $q$ são dois inteiros e $q > 0$ então podemos considerar o valor $p$
dividido por $q$. Esse é um número racional e geralmente é escrito como uma
fração, $\frac{p}{q}$.

\subsection*{Exercício 1}

Como você escreve os seguintes números racionais:

\begin{enumerate}
\item Três pedaços de um bolo dividido em oito pedaços.
\item Uma sacola com $7$ bolas vermelhas de um total de 10 bolas.
\item O número $0.25$.
\item Um copo cheio pela metade de água.
\item A porcentagem $87\%$.
\end{enumerate}

\subsection*{Exercício 2}

\begin{enumerate}
\item Todos os inteiros são racionais? (considere frações onde $q = 1$).

\item Todos os números decimais são racionais? (considere frações com $q = 10,
  100, 1000, \ldots$)

\item Compare as frações $\frac{1}{2}$ e $\frac{5}{10}$. Existe uma única
  maneira de escrever um racional como uma fração?
\end{enumerate}

\subsection*{Dízimas periódicas e fração geratriz}

Quando você realiza a divisão euclidiana de um inteiro $p$ por $q$, o resto é um
inteiro $0 \leq r < q$ e portanto o número de possibilidades é finita. Como
consequência, quando seguimos o algoritmo da divisão para calcular a expansão
decimal de $\frac{q}{q}$, em algum ponto iremos encontrar o mesmo resto que
encontramos em um passo anterior. Continuando com o algoritmo de divisão iremos
obter o mesmo dígito e como consequência obtemos uma dízima periódica.

Como um exemplo, considere $\frac{180}{11}$. As únicas possibilidades para o
resto são $0$, $1$, $2$, $3$, \ldots, $10$ e podemos encontrar rapidamente a
periodicidade como indicado a seguir:
\begin{align*}
  \frac{180}{11} &= 16 + \frac{4}{11} \\
  &= 16 + \frac{1}{10} \frac{40}{11} \\
  &= 16 + \frac{3}{10} + \frac{1}{10} \frac{7}{11} \\
  &= 16 + \frac{3}{10} + \frac{1}{10} \frac{1}{10} \frac{70}{11} \\
  &= 16 + \frac{3}{10} + \frac{6}{100} + \frac{1}{100} \frac{4}{11} \\
  &= 16 + \frac{3}{10} + \frac{6}{100} + \frac{1}{100} \frac{1}{10} \frac{40}{11} \\
  &= 16 + \frac{3}{10} + \frac{6}{100} + \frac{3}{1000} + \frac{1}{1000} \frac{7}{11} \\
  &= 16,36363636\ldots
\end{align*}

De forma parecida, podemos mostrar que qualquer expansão decimal de um número
que é eventualmente periódica é racional. Um método informal para obter a fração
de uma dízima periódica é multiplicar o número por uma potência suficientemente
grande de 10. Por exemplo, $x = 16.363636\ldots$ e $100 x = 1636.3636\ldots$ de
forma que $99 x = 100 x - x = 1636.3636\ldots - 16.363636\ldots = 1620$ e
portanto $x = \frac{1620}{99}$.

\subsection*{Exercício 3}

Utilize o algoritmo da divisão para determinar a expansão decimal das frações:

\begin{enumerate}
\item $\frac{3}{8}$
\item $\frac{2}{3}$
\item $\frac{23}{11}$
\item $\frac{17}{13}$
\end{enumerate}

\subsection*{Exercício 4}

Determine a fração reduzida correspondente as dízimas periódicas abaixo:

\begin{enumerate}
\item $a = 12.350000000000000000\ldots$
\item $b = 0.7777777777777777777\ldots$
\item $c = 3.4545454545454545454\ldots$
\item $d = 14.285714285714285714\ldots$
\end{enumerate}

\section{Potenciação}

\subsection*{Definição}

Se $n > 0$ é um número inteiro e $r$ um número qualquer, escrevemos
$$r^n =
\underset{n\,\text{times}}{\underbrace{r \times r \times \ldots \times r}}.$$

Se $r \neq 0$, podemos mostrar que $r^n \neq 0$ e assim definimos
$$r^{-n} = \frac{1}{r^n}.$$

Também assumimos a convenção de $r^0 = 1$.

\subsection*{Exercício 5}

\begin{enumerate}
\item Determine $2^0$, $2^1$, $2^2$, $2^3$, $2^4$, \ldots, $2^{10}$. Compare com
$0^2$, $2^2$, $3^2$, $4^2$, \ldots, $10^2$.
\item Determine $10^{-1}$, $10^{-2}$, $10^{-3}$, \ldots, $10^{-5}$.
Compare $10^{-100}$ e $\frac{1}{1000000}$.
\end{enumerate}

\subsection*{Exercício 6}

\begin{enumerate}
\item Compare $2^3$ e $3^2$
\item Compare $2^{\left(2^3\right)}$ e ${\left(2^2\right)}^3$
\item Compare $12^2 + 5^2$, $13^2$ e $\left(12+5\right)^2$.
\end{enumerate}

\subsection*{Exercício 7}

Seja $n, m > 0$ inteiros e $r, s$ números quaisquer.

\begin{enumerate}
\item Expresse $r^n \times r^m$ como uma potência de $r$.
\item Expresse $\left(r^{n}\right)^m$ como uma potência de $r$.
\item Expresse $r^n \times s^n$ como uma potência de $r \times s$
\item Assumindo que $r, s \neq 0$ e removendo a suposição de $n, m$ serem
  positivos, mostre que as seguintes afirmações continuam válidas.
\end{enumerate}

\subsection*{Exercício 8}

\begin{enumerate}
\item Determine o quadrado dos números pares
  $0^2$, $2^2$, $4^2$, $6^2$, \ldots O que você nota?
\item Determine o quadrado dos números ímpares
  $1^2$, $3^2$, $5^2$, $7^2$, \ldots O que você nota?
\item Simplifique a fração $\frac{26}{22}$, $\frac{56}{80}$, $\frac{24}{52}$.
  O que você pode dizer sobre a paridade dos numeradores e denominadores das
  frações reduzidas?
\item Seja $p, q > 0$ inteiros tal que
  $\left(\frac{p}{q}\right)^2 = 2$.
  Mostre que $p^2 = 2 q^2$ e deduza que se $p$ é par, então podemos escrever $p
  = 2 r$ para algum inteiro $r$. Deduza $q^2 = 2 r^2$ e a paridade de $q$.
\item Podemos encontrar um racional $x$ tal que $x^2 = 2$?
\end{enumerate}

\section{Solução dos exercícios}

\subsection*{Exercício 1}

\begin{enumerate}
\item $\frac{3}{8}$ de um bolo.
\item $\frac{7}{10}$ bolas são red.
\item $\frac{1}{4}$.
\item A água ocupa $\frac{1}{2}$ do volume do copo.
\item $\frac{87}{100}$ de algo.
\end{enumerate}

\subsection*{Exercício 2}

\begin{enumerate}
\item Se $q = 1$ temos $\frac{p}{q} = p$. Portanto, todos os inteiros são
  racionais.
\item Números decimais são obtidos a partir de um inteiro $p$ movendo a vírgula
  para a esquerda. Portanto, todos os decimais são racionais.
\item Ambos representam $0,5$. Portanto, a forma de representar um racional como
  uma fração não é única.
\end{enumerate}

\subsection*{Exercício 3}

\begin{enumerate}
\item $0.375$
\item $0.666666666666\ldots$
\item $2.090909090909\ldots$
\item $1.3076923076923076923\ldots$
\end{enumerate}

\subsection*{Exercício 4}

\begin{enumerate}
\item $1000 a = 375$ portanto $a = \frac{375}{1000} = \frac{3}{8}$
\item $10b = 7+b$ portanto $b = \frac{7}{9}$
\item $100c = 342 + c$ portanto $c = \frac{342}{99} = \frac{38}{11}$
\item $1000000 d = 14285700 + d$ portanto $d = \frac{14285700}{999999} =
  \frac{100}{7}$
\end{enumerate}

\subsection*{Exercício 5}

\begin{enumerate}
\item Obtemos
  $2^0 = 1$,
  $2^1 = 2$,
  $2^2 = 4$,
  $2^3 = 8$,
  $2^4 = 16$,
  $2^5 = 32$,
  $2^6 = 64$,
  $2^7 = 128$,
  $2^8 = 256$,
  $2^9 = 512$,
  $2^{10} = 1024$ e
  $0^2 = 0$,
  $1^2 = 1$,
  $2^2 = 2$,
  $3^2 = 9$,
  $4^2 = 8$,
  $5^2 = 25$,
  $6^2 = 36$,
  $7^2 = 49$,
  $8^2 = 64$,
  $9^2 = 81$,
  $10^2 = 100$.
  Notamos que $2^n$ é torna-se muito maior que $n^2$ a medida que $n$ cresce.
\item Obtemos
  $10^{-1} = 0.1$,
  $10^{-2} = 0.01$,
  $10^{-3} = 0.001$,
  $10^{-4} = 0.0001$,
  $10^{-5} = 0.00001$.
  $10^{-100}$ teria a mesma forma decimal com cem zeros enquanto 
  $\frac{1}{1000000} = 0.000001$. Notamos que $10^{-n}$ torna-se muito menor que
  $\frac{1}{n}$ a medida que $n$ cresce.
\end{enumerate}

\subsection*{Exercício 6}

\begin{enumerate}
\item $2^3 = 8 \neq 9 = 3^2$
\item $2^{\left(2^3\right)} = 2^8 = 256 \neq 64 = 4^3 = {\left(2^2\right)}^3$
\item $12^2 + 5^2 = 144 + 25 = 169 = 13^2 \neq 17^2 = \left(12+5\right)^2$
\end{enumerate}

\subsection*{Exercício 7}

Seja $n, m > 0$ e $r, s$ números quaisquer.

\begin{enumerate}
\item
$r^n \times r^m =
\underset{n\,\text{times}}{\underbrace{r \times r \times \ldots \times r}} \times
\underset{m\,\text{times}}{\underbrace{r \times r \times \ldots \times r}} =
\underset{n+m\,\text{times}}{\underbrace{r \times r \times \ldots \times r}}
= r^{n+m}$

\item
$\left(r^{n}\right)^m =
\underset{m\,\text{times}}{
\underbrace{{\underset{n\,\text{times}}{\underbrace{r \times r \times \ldots \times r}} \times
\underset{n\,\text{times}}{\underbrace{r \times r \times \ldots \times r}} \times
\ldots \times
\underset{n\,\text{times}}{\underbrace{r \times r \times \ldots \times r}}}}} =
\underset{n \times m\,\text{times}}{\underbrace{r \times r \times \ldots \times r}} =
r^{n \times m}
$

\item $r^n \times s^n =
\underset{n\,\text{times}}{\underbrace{r \times r \times \ldots \times r}} \times
\underset{n\,\text{times}}{\underbrace{s \times s \times \ldots \times s}} =
\underset{n\,\text{times}}{\underbrace{{r \times s} \times {r \times  s} \times \ldots \times {r \times s}}}
= {\left(r \times s\right)}^{n}$

\item Temos
$r^{-n} \times s^{-n} = \frac{1}{r^n} \frac{1}{s^n} =
\frac{1}{r^n \times s^n} = \frac{1}{{r \times s}^{n}} =
{\left(r \times s\right)}^{-n}$. Os outros casos são parecidos, alterando as
várias possibilidades para o sinal.

\end{enumerate}

\subsection*{Exercício 8}

\begin{enumerate}
\item $0$, $4$, $16$, $36$, $64$, $100$, \ldots. O quadrado de um número par é
  par.
\item $1$, $9$, $25$, $49$, $81$, \ldots. O quadrado de um número ímpar é ímpar.
\item $\frac{26}{22} = \frac{13}{11}$,
  $\frac{56}{80} = \frac{28}{40} = \frac{14}{20} = \frac{7}{10}$ e
  $\frac{24}{52} = \frac{12}{26} = \frac{6}{13}$.
  Notamos que o par numerador-denominador das frações reduzidas pode ser
  par-ímpar, ímpar-par, ímpar-ímpar mas não par-par pois nesse caso podemos
  simplificar a fração dividindo o numerador e denominador por 2.
\item $2 = \frac{p}{q} \times \frac{p}{q} = \frac{p \times p}{q \times q}
= \frac{p^2}{q^2}$ e portanto $p^2 = 2 q^2$ é par.
Como uma consequência $p$ é par, e podemos escrever $p = 2r$ para algum inteiro
$r$. Então $2 q^2 = \left(2r\right)^2 = 4 r^2$ e portanto
$q^2 = 2 r^2$ é ímpar. Finalmente, $q$ também é par
\item Se $x^2 = 2$ é racional, podemos escrevê-lo como a fração reduzida
  $x = \frac{p}{q}$. Pela questão anterior, $p, q$ são pares mas vimos que é
  impossível reduzir a fração!
\end{enumerate}

Nota: podemos definir um número real $x > 0$ como $x^2 = 2$ que denotamos por
$\sqrt{2}$. Esse exercício mostra que $\sqrt{2}$ não é racional.
