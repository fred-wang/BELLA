\chapter{Números/Relações (2º Bimestre)}

\section{Polinômio de uma variável}

\subsection*{Definição}

Um polinômio $P$ de uma variável $x$ de grau $n \in \mathbb{N}$ é um objeto da
forma
$$a_n x^n + a_{n-1} x^{n-1} + \ldots + a_1 x + a_0 = \sum_{i=0}^n a_i x^i,$$
onde $a_0$, $a_1$, \ldots, $a_n$ são constantes e $a_n \neq 0$.
Informalmente, um polinômio é uma expressão envolvendo apenas somas e
multiplicação de números que incluem uma variável $x$ representando qualquer
número. Por exemplo, $2 x^2 - 5 x + 1$.

Se $a_1 = $ para $i > n$, o polinômio também pode ser escrito como
$$\sum_{i=0}^n a_i x^i = \sum_{i=0}^{\infty} a_i x^i.$$

Considerando dois polinômios $P = \sum_{i = 0}^{\infty} a_i x^i$ e
$Q = \sum_{i = 0}^{\infty} b_i x^i$, onde $a_i = 0$ para $i$ maior que o grau de
$P$ e $b_i = 0$ para $i$ maior que o grau de $Q$.
A soma, subtração e multiplicação de $P$ e $Q$ são
$$P + Q = \sum_{i = 0}^{\infty} {(a_i + b_i)} x^i,$$
$$P - Q = \sum_{i = 0}^{\infty} {(a_i - b_i)} x^i,$$
$$PQ = \sum_{i = 0}^{\infty} \left(\sum_{j = 0}^{i} a_{j} b_{i - j}\right) x^i.$$

É fácil ver que o coeficiente de $x^i$ é zero se $i$ é maior que os graus de $P$
e $Q$ e então ainda temos um polinômio.

Informalmente, se vemos o polinômio como uma expressão onde a variável $x$ é um
número qualquer, essas são as operações usuais sobre os números.

\subsection*{Exemplos}

A soma de $2x^2 + 7$ e $3x^2 - x$ é $(2+3)x^2 + (0-1)x + (7+0) = 5x^2 - x + 7$.

O subtração de $2Y^2 + 7$ e $3Y^2 - Y$ é $(2Y^2 + 7) - (3Y^2 - Y) = 2Y^2 + 7 - 3Y^2 + Y = -Y^2 + Y + 7$.

A multiplicação de $2a^2 + 7$ e $3a^2 - a$ é
$(2a^2 + 7)(3a^2 - a) = 2a^2 \times 3a^2 + 2a^2 \times -a + 7 \times 3a^2 +
7 \times -a = 6a^4 - 2a^3 + 21a^2 - 7a$.

\subsection*{Exercício 1}

Calcular os seguintes polinômios:
\begin{enumerate}
\item $(z^4 - 2z + 8) + (2z^3 + 6z - 5)$
\item $(2u^3 - 5u + 2) - (6u^7 + u^2 - 2u + 7)$
\item ${(-T^3 + 2T^2 - 5)}{(7T + 1)}$
\item $(x^2 + 1)(x - 2) + (5x+3) - (x^3 - 2x)$
\end{enumerate}

\subsection*{Exercício 2}

Seja $N = \sum_{i=0}^n a_i 10^i$ um inteiro escrito na base 10 (para todo $0
\leq i \leq n$, temos $0 \leq a_i < 9$).

\begin{enumerate}
\item Encontre inteiros $A$ e $B$ tal que
  $$N = 2A  + a_0 = 5B + a_0.$$
  Deduza uma forma de verificar que um número é um múltiplo de $2$ ou $5$.
  O que podemos dizer sobre $N=126743820$?

\item Encontre inteiros $C$ tal que
  $$N = 9C + a_0 + \left(\sum_{i=0}^{n-1} a_{i+1} 10^i\right).$$
  Deduza uma forma de verificar que um número é um múltiplo de $3$ ou $9$.
  O que podemos dizer sobre $N=126743820$?

\item Encontre inteiros $D$ tal que
  $$N = 11D + a_0 + \left(\sum_{i=0}^{n-1} \left(-a_{i+1}\right) 10^i\right).$$
  Deduza uma forma de verificar que o número é um múltiplo de $11$.
  O que podemos dizer sobre $N=126743820$?
\end{enumerate}

\section{Produtos notáveis e Fatoração algébrica}

Vimos como calcular o produto de dois polinômios $P$ e $Q$, $PQ$. Também podemos
fazer a operação inversa, i.e., fatorar um polinômio $R$ como o produto de dois
polinômios, $R = PQ$.

De maneira geral, podemos desmembrar o produto de expressões algébricas e
deduzir a fatoração. Alguns produtos notáveis.

Factor comum: $c(a + b) = ca + cb$.

\begin{center}
\begin{tikzpicture}
  \draw (0,0) rectangle(7,3)
              rectangle(2,0);
  \path (1,1) node (a1) {$ac$};
  \path (3.5,1) node (a2) {$bc$};
  \path (7.5,1.5) node (c1) {$c$};
  \path (1,3.5) node (c2) {$a$};
  \path (4.5,3.5) node (c3) {$b$};
\end{tikzpicture}
\end{center}

Quadrado de um binômio:
$${(a+b)}^2 = a^2 + 2ab + b^2,$$
$${(a-b)}^2 = a^2 - 2ab + b^2.$$

\begin{center}
\begin{tikzpicture}
  \draw (0,0) rectangle(4,4)
              rectangle(1,1)
              rectangle(0,0);
  \path (2,2) node (a1) {$b^2$};
  \path (.5,.5) node (a2) {$a^2$};
  \path (.5,2) node (a3) {$ab$};
  \path (2,.5) node (a4) {$ab$};
  \path (.5,4.5) node (c1) {$a$};
  \path (4.5,3) node (c2) {$b$};
  \path (4.5,.5) node (c3) {$a$};
  \path (3,4.5) node (c4) {$b$};
\end{tikzpicture}
\end{center}

Produto de dois binômios conjugados:
$$(a + b)(a - b) = a^2 - b^2.$$

Cubo de um binômio:
$${(a+b)}^3  = a^3 + 3a^2b + 3ab^2 + b^3,$$
$${(a-b)}^3  = a^3 - 3a^2b + 3ab^2 - b^3.$$

\subsection*{Exemplos}

Podemos fatorar:
\begin{itemize}
  \item $3x^3 + 2x^2 + 2x - 7 = (x-1)(3x^2+5x+7)$.
  \item $(x-1)^2 = (x-1)(x-1) = x^2 - x - x + 1 = x^2-2x+1$
  \item $2^3 = {(1+1)}^3 = 1 + 3 + 3 + 1 = 8 = 2 \times 2 \times 2$
\end{itemize}

\subsection*{Exercício 3}

Expanda os polinômios:

\begin{enumerate}
\item $(7+3a)^2$
\item $(2x-3)(2x+3)$
\item $(2T-1)^3$
\end{enumerate}

\subsection*{Exercício 4}

Fatore os polinômios:

\begin{enumerate}
\item $21x^2 - 12x$
\item $3a^2 + a-2$ (utilize $a+1$ como fator comum)
\item $4z^2 - 4z + 1$ (quadrado de um binômio)
\item $T^2 - 49$ (binômios conjugados)
\item $x^3 + 3x^2 + 3x + 1$ (cubo de um binômio)
\end{enumerate}

\section{Problemas com áreas e volumes}

\subsection*{Exercício 5}

João possui uma caixa cúbica de lado $3+x$cm. José possui uma caixa cúbica de
lado $3$cm e uma outra caixa com base quadrada de lado $x+5$cm e altura $x$cm.
Expresse o volume de cada uma das caixa em função de $x$. Qual caixa possui o
mair volume para $x=1$cm e $x = 3$cm? Para qualquer valor de $x > 0$ João e José
possuem o mesmo espaço?

\subsection*{Exercício 6 (formato de papel)}

\begin{center}
\begin{tikzpicture}
  \draw (0,0) rectangle(2.1,2.97)
              rectangle(4.2,0);
  \path (1,1.5) node (a1) {$A_5$};
  \path (3.1,1.5) node (a2) {$A_5$};
  \path (4.5,3.5) node (a3) {$A_4$};
  \path (5,1.5) node (c1) {$21\text{cm}$};
\end{tikzpicture}
\end{center}

O lado menor de uma folha A4 e o lado maior de uma folha A5 possuem $21$cm. Além
disso, ambas as folhas possuem a mesma proporção
$$k = \frac{\text{lado maior}}{\text{lado menor}}i.$$. Se cortamos uma folha A4
na sua metada maior, obtemos duas folhas A5.

\begin{enumerate}
\item Expresse o lado maior de uma folha A4 em função do lado menor de uma folha
  A5.
\item Expresse o lado maior de uma folha A4 e o lado menor de uma folha A5 em
  função de $k$.
\item Determine $k$
\item Qual é o tamanho de uma folha A4?
\end{enumerate}

\subsection*{Exercício 7 (número áureo)}

\begin{center}
\begin{tikzpicture}
  \draw (0,0) rectangle(5,3.09)
              rectangle(1.91,0);
  \path (1,3.5) node (c1) {$a$};
  \path (4,3.5) node (c2) {$b$};
  \path (5.5,1.5) node (c3) {$b$};

\end{tikzpicture}
\end{center}

Consideremos um retângulo de lado menor $a$, de lado maior $b$ e de
proporção $R = b/a$. Expresse a proporção do retângulo de lado menor $b$ e de
lado maior $a+b$ em função de $1/R$. Se os dois retângulos possuem a mesma
proporção, mostrar que $R^2 - R - 1 = 0$.

Expanda o produto seguinte e deduza $R$:
$$\left(2x-\sqrt{5}-1\right) \left(2x+\sqrt{5}-1\right).$$

\section{Solução dos exercícios}

\subsection*{Exercício 1}

\begin{enumerate}
\item $z^4 + 2z^3 + 4z + 3$
\item $-6u^7 + 2u^3 -u^2 - 3u - 5$
\item $-7T^4 + 13T^3 + 2T^2 -35T -5$
\item $-2x^2 + 8x + 1$
\end{enumerate}

\subsection*{Exercício 2}

\begin{enumerate}
\item $A = \sum_{i=0}^{n-1} {5 \times a_{i+1} \times 10^i}$ e
  $B = \sum_{i=0}^{n-1} {2 \times a_{i+1} \times 10^i}$. Como uma consequência,
  nos apenas precisamos verificar que o último dígito $a_0$ é um múltiplo de $2$
  (isso é, $0$, $2$, $4$, $6$ e $8$) ou múltiplo de $5$ (isso é, $0$ e $5$).
  O último dígito de $N=126743820$ é $0$ e portanto $N$ é múltiplo de $2$ e $5$.

\item $C =\sum_{i=0}^{n-1} {a_{i+1} 10^i}$. Como uma consequência, é suficiente
  verificar que $a_0 + \left(\sum_{i=0}^{n-1} a_{i+1} 10^i\right)$ é um múltiplo
  de $3$ ou $9$. Podemos repetir esse método: é suficiente verificar que
  $a_0 + a_1 + \left(\sum_{i=0}^{n-2} a_{i+2} 10^i\right)$ é um múltiplo $3$ ou
  $9$, onde $a_0 + a_1 + a_2 + \left(\sum_{i=0}^{n-3} a_{i+3} 10^i\right)$ é um
  múltiplo de $3$ ou $9$ etc. Finalmente, é suficiente verificar que se a soma
  dos dígitos de $N$, $\sum_{i=0}^n a_i$, é um múltiplo de $3$ ou $9$. Logo, para
  $N=126743820$ temos $1+2+6+7+4+3+8+2+0=36$ e $3+6=9$ é um múltiplo de $9$ mas
  não $3$. O mesmo é válido para $N$.

\item $D = \sum_{i=0}^{n-1} \left(a_{i+1}\right) 10^i$. Como uma consequência, é
  suficiente verificar que
  $a_0 + \left(\sum_{i=0}^{n-1} \left(-a_{i+1}\right) 10^i\right)$ é múltiplo de
  $11$, onde
  $a_0 - a_1 + \left(\sum_{i=0}^{n-2} a_{i+2} 10^i\right)$ é um múltiplo de
  $11$, onde
  $a_0 - a_1 + a_2 + \left(\sum_{i=0}^{n-2} \left(-a_{i+2}\right) 10^i\right)$ é
  um múltiplo de $11$ etc. Finalmente, apenas precisamos verificar se a soma
  alternada dos dígitos de $N$,
  $\sum_{i=0}^n {\left(-1\right)^{i} a_i^i}$, é multiplo de $11$. Logo,
  para $N=126743820$ temos $1-2+6-7+4-3+8-2+0 = 5$ que não é multiplo de $11$.
\end{enumerate}

\subsection*{Exercício 3}

\begin{enumerate}
\item $9a^2 + 42a + 49$
\item $4x^2-9$
\item $8T^3-12T^2+6T-1$
\end{enumerate}

\subsection*{Exercício 4}

\begin{enumerate}
\item $3x (7x - 4)$
\item $(a+1)(3a-2)$
\item $(2z-1)^2$
\item $(T+7)(T-7)$
\item $(x+1)^3$
\end{enumerate}

\subsection*{Exercício 5}

O volume da caixa de João é $(3+x)^3 = x^3+9x^2+27x+27$cm³. Os volumes das
caixas de José são $3^3=27$cm³ e $x(x+5)^2 = x^3+10x^2+25x$cm³ e portanto o
espaço total é $x^3+10x^2+25x + 27$cm³.

Para $x=1$cm encontramos $5^3 = 125$cm³ para João e $6^2 + 27 = 63$cm³ para
José. Ao contrário, si $x=3$cm encontramos $6^3 = 216$cm³ para João e
$6\times11^2 + 27 = 753$cm³ para José.

Se João e José possuem o mesmo espaço,
$x^3+9x^2+27x+27 = x^3+10x^2+25x + 27$ e portanto
$x^2 = 2x$. Finalmente, $x(x-2) = 0$ de modo que as soluções são
$x=0$cm ou $x = 2$cm. Como estamos interessados em $x > 0$ apenas a resposta é
$x = 2$cm.

\subsection*{Exercício 6}

\begin{enumerate}
\item O lado maior da uma folha A4 é o dobro do lado menor de uma folha A5.
\item $21k$ e $21/k$.
\item $21k = 2 (21/k)$ e portanto $k^2 = 2$ e $k = \sqrt{2}$.
\item O lado menor mede $21$cm e o lado maior $21k = 21\sqrt{2} \approx
  29,7\text{cm}$
\end{enumerate}

\subsubsection*{Exercício 7}

$\frac{a+b}{b} = \frac{a}{b} + 1  = \frac{1}{R} + 1$. Se
$R = \frac{1}{R} + 1$, $R^2 - R - 1 = 0$.

Obtemos $4x^2-4x-4 = 4(x^2-x-1)$.

Então, $R^2 - R - 1 = 0$ se e somente se
$2R - \sqrt{5} - 1 = 0$ ou $2R + \sqrt{5} - 1 = 0$ se e somente se
$R = \frac{1 - \sqrt{5}}{2} \approx -0.618$ ou
$R = \frac{1 + \sqrt{5}}{2} \approx 1.618$. Porque $R > 0$,
a única possibilidade é $R = \frac{1 + \sqrt{5}}{2}$.
